\section{Mathematica 8 implementáció}
Az alábbi implementáció Mathematica 8 (vagy annál újabb) környezethez készült, anélkül nem használható.
Letölthető és installálható \cite{Nag11}-ben leírt módon.
Permutációcsoportok kezelésére pár függvény már alapból benne van Mathematica-ban,
definiálni tudunk csoportokat, pár alap csoport előre benne van
(szimmetrikus, alternáló, ciklikus, diéder és ábel-csoportok, valamint a sporadikus csoportok nagy része).
Le tudjuk kérdezni egy csoport erős generátorrendszerét (Schreier-Sims algoritmusnak egy változata bele van írva),
valamint egy elem centralizátorát.
Ezeken kívül a többi algoritmust az én implementációm biztosítja.
A függelék innentől angol nyelven folytatódik.
\clearpage

\subsection{License}
This package can be used under the rules of the 2-clause BSD license:
\lstset{
	breaklines=true,
	showstringspaces=false,
}
\begin{lstlisting}
Copyright (c) 2011, Gergely Nagy
All rights reserved.

Redistribution and use in source and binary forms, with or without modification, are permitted provided that the following conditions are met:

- Redistributions of source code must retain the above copyright notice, this list of conditions and the following disclaimer.
- Redistributions in binary form must reproduce the above copyright notice, this list of conditions and the following disclaimer in the documentation and/or other materials provided with the distribution.

THIS SOFTWARE IS PROVIDED BY THE COPYRIGHT HOLDERS AND CONTRIBUTORS "AS IS" AND ANY EXPRESS OR IMPLIED WARRANTIES, INCLUDING, BUT NOT LIMITED TO, THE IMPLIED WARRANTIES OF MERCHANTABILITY AND FITNESS FOR A PARTICULAR PURPOSE ARE DISCLAIMED. IN NO EVENT SHALL THE COPYRIGHT HOLDER OR CONTRIBUTORS BE LIABLE FOR ANY DIRECT, INDIRECT, INCIDENTAL, SPECIAL, EXEMPLARY, OR CONSEQUENTIAL DAMAGES (INCLUDING, BUT NOT LIMITED TO, PROCUREMENT OF SUBSTITUTE GOODS OR SERVICES; LOSS OF USE, DATA, OR PROFITS; OR BUSINESS INTERRUPTION) HOWEVER CAUSED AND ON ANY THEORY OF LIABILITY, WHETHER IN CONTRACT, STRICT LIABILITY, OR TORT (INCLUDING NEGLIGENCE OR OTHERWISE) ARISING IN ANY WAY OUT OF THE USE OF THIS SOFTWARE, EVEN IF ADVISED OF THE POSSIBILITY OF SUCH DAMAGE.
\end{lstlisting}
\clearpage

\subsection{Documentation}
\lstset{
	language=Mathematica,
	breaklines=true,
	showstringspaces=false,
	basicstyle=\footnotesize,
	postbreak={\space\space\space\space\space\space\space\space\space\space},
	breakindent=0pt,
	breakautoindent=false,
}

\begin{itemize}

\item NullQ[expr] gives True if expr is Null, and False otherwise.
\begin{itemize}
\item Test whether Null is Null:
\begin{lstlisting}
 In[1]:= NullQ[Null]
Out[1]:= True
\end{lstlisting}
\item Test whether "Null" is Null:
\begin{lstlisting}
 In[2]:= NullQ["Null"]
Out[2]:= False
\end{lstlisting}
\end{itemize}

\item GroupQ[expr] gives True if expr is a group, and False otherwise.
\begin{itemize}
\item Test whether SymmetricGroup[3] is a group:
\begin{lstlisting}
 In[1]:= GroupQ[SymmetricGroup[3]]
Out[1]:= True
\end{lstlisting}
\end{itemize}

\item GroupActionSetSort[actset] sorts the elements of actset into an order in which elements of option GroupActionBase are the first ones and then other elements follow.
\begin{itemize}
\item If there are no base given, it is just a normal sort:
\begin{lstlisting}
 In[1]:= GroupActionSetSort[{3,1,5,2,7,9,6}]
Out[1]:= {1, 2, 3, 5, 6, 7, 9}
\end{lstlisting}
\item If base is given, then base elements come first:
\begin{lstlisting}
 In[2]:= GroupActionSetSort[{3,1,5,2,7,9,6}, GroupActionBase -> {2,7,4,6}]
Out[2]:= {2, 7, 6, 1, 3, 5, 9}
\end{lstlisting}
\end{itemize}

\item CyclesActionSet[elem] gives the action set of a group element.
\begin{itemize}
\item Action set for Cycles[{{1,3},{2,7}}]:
\begin{lstlisting}
 In[1]:= CyclesActionSet[Cycles[{{1,3},{2,7}}]]
Out[1]:= {1, 2, 3, 7}
\end{lstlisting}
\item We can use GroupActionBase to specify order:
\begin{lstlisting}
 In[2]:= CyclesActionSet[Cycles[{{1,3},{2,7}}], GroupActionBase -> {1,7,5,3}]
Out[2]:= {1, 7, 3, 2}
\end{lstlisting}
\end{itemize}

\item GroupActionSet[group] gives the action set of a group.
\begin{itemize}
\item Action set for a group:
\begin{lstlisting}
 In[1]:= GroupActionSet[MathieuGroupM11[]]
Out[1]:= {1, 2, 3, 4, 5, 6, 7, 8, 9, 10, 11}
\end{lstlisting}
\item We can use GroupActionBase to specify order:
\begin{lstlisting}
 In[2]:= GroupActionSet[SymmetricGroup[4], GroupActionBase -> {3,2}]
Out[2]:= {3, 2, 1, 4}
\end{lstlisting}
\end{itemize}

\item GroupExponent[group] gives the exponent of the group.
\begin{itemize}
\item Compute the exponent of a group:
\begin{lstlisting}
 In[1]:= GroupExponent[DihedralGroup[8]]
Out[1]:= 8
\end{lstlisting}
\item The exponent of a group is always a divisor of its order:
\begin{lstlisting}
 In[2]:= GroupOrder[DihedralGroup[8]]/GroupExponent[DihedralGroup[8]]
Out[2]:= 2
\end{lstlisting}
\end{itemize}

\item GroupElementFromImage[group, a, b] gives an element of the group which moves a to b, or Null if there is no such element.
\begin{itemize}
\item We can get an element of CyclicGroup[5] which moves 1 to 3:
\begin{lstlisting}
 In[1]:= GroupElementFromImage[CyclicGroup[5], 1, 3]
Out[1]:= Cycles[{{1, 3, 5, 2, 4}}]
\end{lstlisting}
\item If there are no such element it gives Null:
\begin{lstlisting}
 In[2]:= NullQ[GroupElementFromImage[CyclicGroup[5], 1, 6]]
Out[2]:= True
\end{lstlisting}
\end{itemize}

\item GroupIrredundantStabilizerChain[group] gives a stabilizer chain of the group according to the option GroupActionBase, but skips redundant base elements.
\begin{itemize}
\item The built-in GroupStabilizerChain function can give redundant stabilizer chain:
\begin{lstlisting}
 In[1]:= GroupStabilizerChain[CyclicGroup[5], GroupActionBase -> {1, 3}]
Out[1]:= {{} -> PermutationGroup[{Cycles[{{1, 2, 3, 4, 5}}]}], {1} -> PermutationGroup[{}], {1, 3} -> PermutationGroup[{}]}
\end{lstlisting}
\item However, GroupIrredundantStabilizerChain always gives irredundant:
\begin{lstlisting}
 In[2]:= GroupIrredundantStabilizerChain[CyclicGroup[5], GroupActionBase -> {1, 3}]
Out[2]:= {{} -> PermutationGroup[{Cycles[{{1, 2, 3, 4, 5}}]}], {1} -> PermutationGroup[{}]}
\end{lstlisting}
\end{itemize}

\item GroupConjugatesQ[group, elem1, elem2] gives True if elem1 and elem2 are conjugates in the group, and False otherwise.
\begin{itemize}
\item Cycles[{{1,2,3}}] and Cycles[{{1,3,2}}] are conjugates in SymmetricGroup[4]:
\begin{lstlisting}
 In[1]:= GroupConjugatesQ[SymmetricGroup[4], Cycles[{{1,2,3}}], Cycles[{{1,3,2}}]]
Out[1]:= True
\end{lstlisting}
\item But they are not conjugates in AlternatingGroup[4]:
\begin{lstlisting}
 In[2]:= GroupConjugatesQ[AlternatingGroup[4], Cycles[{{1,2,3}}], Cycles[{{1,3,2}}]]
Out[2]:= False
\end{lstlisting}
\item We can check conjugacy not just for group elements, in that case it gives True iff there is a group element that conjugates elem1 to elem2:
\begin{lstlisting}
 In[3]:= GroupConjugatesQ[SymmetricGroup[4], Cycles[{{1,2,3},{5,6}}], Cycles[{{1,3,2},{5,6}}]]
Out[3]:= True
\end{lstlisting}
\end{itemize}

\item GroupConjugacyClassRepresentatives[group] gives a list of group elements which represent the conjugacy classes.
\begin{itemize}
\item We can get class representatives for a group:
\begin{lstlisting}
 In[1]:= GroupConjugacyClassRepresentatives[AlternatingGroup[4]]
Out[1]:= {Cycles[{}], Cycles[{{1, 4}, {2, 3}}], Cycles[{{1, 3, 2}}], Cycles[{{1, 3, 4}}]}
\end{lstlisting}
\end{itemize}

\item GroupNumOfConjugacyClasses[group] gives the number of conjugacy classes in the group.
\begin{itemize}
\item The number of conjugacy classes in SymmetricGroup[n] is always PartitionsP[n]:
\begin{lstlisting}
 In[1]:= Table[GroupNumOfConjugacyClasses[SymmetricGroup[n]], {n, 6}]
Out[1]:= {1, 2, 3, 5, 7, 11}
 In[2]:= Table[PartitionsP[n], {n, 6}]
Out[2]:= {1, 2, 3, 5, 7, 11}
\end{lstlisting}
\end{itemize}

\item GroupConjugacyClassSizes[group] gives the list of sizes of the conjugacy classes (in the same order as GroupConjugacyClassRepresentatives[group] gives the elements.
\begin{itemize}
\item We can get sizes of conjugacy classes:
\begin{lstlisting}
 In[1]:= GroupConjugacyClassSizes[AlternatingGroup[4]]
Out[1]:= {1, 3, 4, 4}
\end{lstlisting}
\item The ordering corresponds to GroupConjugacyClassRepresentatives:
\begin{lstlisting}
 In[2]:= GroupConjugacyClassRepresentatives[AlternatingGroup[4]]
Out[2]:= {Cycles[{}], Cycles[{{1, 4}, {2, 3}}], Cycles[{{1, 3, 2}}], Cycles[{{1, 3, 4}}]}
\end{lstlisting}
\end{itemize}

\item GroupConjugacyClassInverses[group] gives the list whose k-th element is the index of the conjugacy class in which the inverses of the elements of the k-th conjugacy class are.
\begin{itemize}
\item We can get indices of inverse conjugacy classes:
\begin{lstlisting}
 In[1]:= GroupConjugacyClassInverses[AlternatingGroup[4]]
Out[1]:= {1, 2, 4, 3}
\end{lstlisting}
\item The ordering corresponds to GroupConjugacyClassRepresentatives:
\begin{lstlisting}
 In[2]:= GroupConjugacyClassRepresentatives[AlternatingGroup[4]]
Out[2]:= {Cycles[{}], Cycles[{{1, 4}, {2, 3}}], Cycles[{{1, 3, 2}}], Cycles[{{1, 3, 4}}]}
\end{lstlisting}
\end{itemize}

\item GroupConjugacyClassNum[group, elem] gives the index of the conjugacy class of elem in group.
\begin{itemize}
\item We can get index of the conjugacy class of an element:
\begin{lstlisting}
 In[1]:= GroupConjugacyClassNum[AlternatingGroup[4], Cycles[{{1,4,2}}]]
Out[1]:= 4
\end{lstlisting}
\item The index corresponds to GroupConjugacyClassRepresentatives:
\begin{lstlisting}
 In[2]:= GroupConjugacyClassRepresentatives[AlternatingGroup[4]]
Out[2]:= {Cycles[{}], Cycles[{{1, 4}, {2, 3}}], Cycles[{{1, 3, 2}}], Cycles[{{1, 3, 4}}]}
\end{lstlisting}
\end{itemize}

\item GroupConjugacyClass[group, n] gives the full list of elements in the n-th conjugacy class.
\begin{itemize}
\item We can get a full conjugacy class:
\begin{lstlisting}
 In[1]:= GroupConjugacyClass[AlternatingGroup[4], 4]
Out[1]:= {Cycles[{{1, 4, 2}}], Cycles[{{2, 4, 3}}], Cycles[{{1, 2, 3}}], Cycles[{{1, 3, 4}}]}
\end{lstlisting}
\item The ordering corresponds to GroupConjugacyClassRepresentatives:
\begin{lstlisting}
 In[2]:= GroupConjugacyClassRepresentatives[AlternatingGroup[4]]
Out[2]:= {Cycles[{}], Cycles[{{1, 4}, {2, 3}}], Cycles[{{1, 3, 2}}], Cycles[{{1, 3, 4}}]}
\end{lstlisting}
\end{itemize}

\item GroupCharacterScalarProduct[group, chi, psi] gives the scalar product of two characters (chi and psi) of the group given by a list of values in the conjugacy classes.
\begin{itemize}
\item We can compute scalar product of two characters:
\begin{lstlisting}
 In[1]:= GroupCharacterScalarProduct[AlternatingGroup[4], {1, 1, 1, 1}, {4, 0, 0, 0}]
Out[1]:= 1/3
\end{lstlisting}
\item Irreducible characters form an ortonormal base in the space of characters:
\begin{lstlisting}
 In[2]:= g = AlternatingGroup[4]; tbl = GroupCharacterTable[g]; FullSimplify[Table[GroupCharacterScalarProduct[g, tbl[[i]], tbl[[j]]], {i, Length[tbl]}, {j, Length[tbl]}]]
Out[2]:= {{1, 0, 0, 0}, {0, 1, 0, 0}, {0, 0, 1, 0}, {0, 0, 0, 1}}
\end{lstlisting}
\item We can compute it over a finite field:
\begin{lstlisting}
 In[2]:= GroupCharacterScalarProduct[AlternatingGroup[4], {1, 1, 1, 1}, {4, 0, 0, 0}, Modulus -> 7]
Out[2]:= 5
\end{lstlisting}
\end{itemize}

\item GroupDixonPrime[group] gives the smallest prime number (p) such that GF[p] can be used to represent all the complex characters in.
\begin{itemize}
\item It gives the smallest p prime of the form k*GroupExponent[group]+1 such that p > 2*Sqrt[GroupOrder[group]]:
\begin{lstlisting}
 In[1]:= GroupExponent[MathieuGroupM11[]]
Out[1]:= 1320
 In[2]:= N[2*Sqrt[GroupOrder[MathieuGroupM11[]]]]
Out[2]:= 177.989
 In[3]:= GroupDixonPrime[MathieuGroupM11[]]
Out[3]:= 1321
\end{lstlisting}
\item Another example:
\begin{lstlisting}
 In[4]:= GroupDixonPrime[CyclicGroup[7]]
Out[4]:= 29
\end{lstlisting}
\end{itemize}

\item GroupCharacterTableOverFiniteField[group] gives the character table of the group over GF[p] where p is given by GroupDixonPrime[group].
\begin{itemize}
\item We can calculate the character table of a group over GF[GroupDixonPrime[group]]:
\begin{lstlisting}
 In[1]:= GroupDixonPrime[AlternatingGroup[4]]
Out[1]:= 7
 In[2]:= MatrixForm[GroupCharacterTableOverFiniteField[AlternatingGroup[4]]]
Out[2]:= {{1, 1, 1, 1}, {1, 1, 2, 4}, {1, 1, 4, 2}, {3, 6, 0, 0}}
 In[3]:= MatrixForm[GroupCharacterTable[AlternatingGroup[4]]]
Out[3]:= {{1, 1, 1, 1}, {1, 1, (-1)^(2/3), -(-1)^(1/3)}, {1, 1, -(-1)^(1/3), (-1)^(2/3)}, {3, -1, 0, 0}}
\end{lstlisting}
\item The ordering of the columns corresponds to GroupConjugacyClassRepresentatives:
\begin{lstlisting}
 In[4]:= GroupConjugacyClassRepresentatives[AlternatingGroup[4]]
Out[4]:= {Cycles[{}], Cycles[{{1, 4}, {2, 3}}], Cycles[{{1, 3, 2}}], Cycles[{{1, 3, 4}}]}
\end{lstlisting}
\end{itemize}

\item GroupCharacterTable[group] gives the character table of the group.
\begin{itemize}
\item We can calculate the character table of a group:
\begin{lstlisting}
 In[1]:= GroupCharacterTable[DihedralGroup[5]]
Out[1]:= {{1, 1, 1, 1}, {1, 1, 1, -1}, {2, (-1 - Sqrt[5])/2, (-1 + Sqrt[5])/2, 0}, {2, (-1 + Sqrt[5])/2, (-1 - Sqrt[5])/2, 0}}
\end{lstlisting}
\item The ordering of the columns corresponds to GroupConjugacyClassRepresentatives:
\begin{lstlisting}
 In[2]:= GroupConjugacyClassRepresentatives[DihedralGroup[5]]
Out[2]:= {Cycles[{}], Cycles[{{1, 2, 3, 4, 5}}], Cycles[{{1, 3, 5, 2, 4}}], Cycles[{{1, 3}, {4, 5}}]}
\end{lstlisting}
\end{itemize}

\end{itemize}

\clearpage
\subsection{Source code}
\lstset{
	language=Mathematica,
	tabsize=2,
	breaklines=true,
	showstringspaces=false,
	basicstyle=\footnotesize,
	numbers=left,
	numberstyle=\footnotesize,
	postbreak={},
	breakindent=20pt,
	breakautoindent=true,
}
\lstinputlisting{../GroupExt/GroupExt/GroupExt.m}
