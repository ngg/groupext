\section{Black-box csoportok}
\label{sec:blackbox}
A black-box csoportok eredeti ötlete és definíciója magyar matematikusoktól származik,
Babai László és Szemerédi Endre publikálta \cite{BS84}-ben.
Ma a legtöbb helyen inkább kicsit általánosabb definíciókat használnak, amivel kevésbé körülményes az
egyes csoportok tényleges leírása. Az itt leírt definíciót \cite{Ser03}-ból vettem.

A black-box csoport egy olyan csoport,
aminek elemeit a véges $\Sigma$ ábécé feletti legfeljebb $N$ hosszú szavaival azonosítjuk.
Nincs megkövetelve, hogy egy elemnek csak egy szó felelhessen meg, se az hogy minden szó hozzátartozzon
egy elemhez.
A csoportműveleteket egy orákulum végzi.
Ha adott két szó, amik a $g, h\in G$ elemeket reprezentálják,
akkor meg tudjuk állapítani, hogy $g=1$ igaz-e,
valamint ki tudjuk számolni a $g^{-1}$-hez és a $gh$-hoz tartozó szavakat.
Általában a csoport megadása egy generátorrendszer segítségével történik, az elemeihez tartozó szavak megadásával.

Előfordulhat, hogy az orákulum egy $\overline{G} \ge G$ nagyobb csoport elemeihez tartozó szavakat fogad el,
ilyenkor csak azt tesszük fel, hogy azt tudjuk megállapítani, hogy a szó $\overline{G}$-beli-e,
azt nem, hogy $G$ vagy $\overline{G}\setminus G$ eleme.

Erre példa egy véges $F$ test feletti $G$ $n\times n$-es mátrixcsoport.
$\Sigma=F$ adja az ábécét, a szavak $n^2$ hosszúak,
valamint $\overline{G}$ az összes invertálható $F$ feletti $n\times n$-es mátrix csoportja.
Természetesen a permutációcsoportok, illetve a policiklikus csoportok is könnyen leírhatók black-box csoportokként.

A black-box csoportokra vonatkozó algoritmusoknak  egyik leggyakoribb típusa az általunk később is sokat használt
orbit-számoló algoritmusok, amik közül párat az \ref{subsec:orbit}. alfejezetben részletezünk.
A másik általunk itt leírt algoritmus konjugáltosztályokra tudja bontani a csoportot, ami a reprezentációelméletben nagyon fontos,
erről szól az \ref{subsec:konj}. alfejezet.

\subsection{Orbit-algoritmusok}
\label{subsec:orbit}
Gyakori eset, hogy $G$ hatását nézzük egy $\Omega$ halmazon, és ott egy $\alpha\in\Omega$ elem orbitját,
vagyis $\alpha^G = \{ \alpha^g \mid g \in G \}$-t keressük. Ilyen előfordulhat, ha $G \le \Sym(\Omega)$,
de például az is ide tartozik, ha $g \in G$ konjugáltosztályát szeretnénk meghatározni, ugyanis választhatjuk
$\Omega=G$-t és hatásnak a konjugálást tekinthetjük.

Feltesszük, hogy adott $\beta\in\Omega$ és $g\in G$-re meg tudjuk határozni a $\beta^g$ képet, valamint $\Omega$-beli
elemeket össze tudunk hasonlítani egymással. Lehetséges, hogy $\Omega$ nagyon nagy, így nem tesszük fel, hogy fel tudjuk sorolni az elemeit.

Legyen $G=\langle S\rangle$. $\alpha^G$ az a legszűkebb részhalmaza $\Omega$-nak, ami tartalmazza $\alpha$-t, valamint zárt az $S$-beli csoportelemek hatására nézve.
Vegyük azt az irányított gráfot, aminek csúcsai $\Omega$ elemei és $\beta$-ból $\gamma$-ba akkor megy él, ha van olyan $S$-beli elem, ami $\beta$-t $\gamma$-ba viszi.
Ebben a gráfban kell megkeresnünk az $\alpha$-t tartalmazó összefüggő komponenst.
Mivel $G$ véges rendű, így $\beta^g=\gamma$ esetén létezik $l$, amire $\gamma^{g^l}=\beta$, vagyis az összefüggő komponensek erősen osszefüggők.
A komponenst meg tudjuk találni szélességi kereséssel $\mathcal{O}(|S| |\alpha^G|)$ idő alatt, $\mathcal{O}(|\alpha^G|)$ memória használatával.

Most nézzünk egy általánosabb verziót.
Legyen adott $\Omega$-n egy algebrai struktúra, és tegyük fel, hogy $G$ hatása olyan, hogy az minden $*$ $\Omega$-beli műveletre nézve disztribuív,
vagyis $(\alpha * \beta)^g = \alpha^g * \beta^g$.
$A \subseteq \Omega$-ra szeretnénk kiszámolni $\langle A^G \rangle$-t, vagyis
$\Omega$-nak azt a legszűkebb részhalmazát, ami tartalmazza $A$ minden elemét és zárt $G$ hatására és az $\Omega$-beli művelet(ek)re is.
Ha az előző konjugálós példát nézzük úgy, hogy $\Omega$-n is nézzük a csoportműveletet, akkor ezzel $A$ normális lezártját szeretnénk meghatározni.
Ha $G'=[G,G]$ kommutátor részcsoportot szeretnénk meghatározni,
akkor azt így tudjuk megtenni, hiszen $G'=\langle[s,t] \mid s,t \in S\rangle_N$, vagyis az $S$-beli elemek kommutátorainak normális lezártja.
Ezzel tudunk feloldhatóságot valamint nilpotenciát is ellenőrizni.

Mivel a végeredmény zárt az $\Omega$-beli művelet(ek)re, így célszerű az algoritmusnak csak egy generátorrendszert előállítani.
Feltesszük, hogy tudjuk ellenőrizni, hogy $\Omega$ egy eleme benne van-e pár $\Omega$-beli elem által generált struktúrában.
Az algoritmus az előzőnek egy egyszerű változtatásával kapható, egyrészt az elején a szélességi bejárást $A$ minden eleméből egyszerre kell indítani,
másrészt ha egy új csúcshoz érünk, akkor csak akkor kell felírnunk a generátorelemek listájába, ha az addigiak által generált struktúrában nincs benne.
Érdemes itt megjegyeznünk, hogy ha $\Omega$-n legalább egy csoportstruktúra van, akkor minden új generátorelem a csoport elemszámát legalább megduplázza,
vagyis legfeljebb $\mathcal{O}(\log|\langle A^G\rangle |)$ generátorelemet írunk fel. A számítási idő nagyrésze jellemzően arra a számolásra megy el, amikor egy elemről
megpróbáljuk eldönteni, hogy benne van-e a már felírt elemek által generált struktúrában. Vannak olyan randomizált algoritmusok is erre a célra, amikhez nem szükséges feltennünk,
hogy ezt el tudjuk dönteni, ezekről \cite{Ser03}-ban olvashatunk.

\subsection{Konjugáltosztályok megtalálása}
\label{subsec:konj}
Tegyük fel, hogy a $g, h \in G$ csoportelemekről valahogy el tudjuk dönteni, hogy konjugáltak-e, meg tudjuk határozni $C_G(g)$-t, vagyis a $g$ elem centralizátorát, valamint,
hogy tudunk részcsoportokban (közel) egyenletes eloszlású véletlen elemet kiválasztani.
Ezeket permutációcsoportoknál meg fogjuk mutatni, hogyan tudjuk megcsinálni.
Feladatunk, hogy megtaláljuk a konjugáltosztályokat a csoportban, vagyis az osztályoknak megtaláljuk egy-egy reprezentáns elemét.
Bár sok mindent felhasználunk, amit csak speciális csoportoknál tudunk megtenni, mégis az osztályok megtalálása minden csoport esetén ugyanazzal az algoritmussal történhet,
azért írom a black-box csoportok fejezetében.

A legegyszerűbb algoritmus annyiból áll csak, hogy sorban választunk véletlenszerű elemeket a csoportból, megnézzük, hogy konjugáltak-e az eddig megtalált reprezentáns elemekkel,
és ha egyikkel sem az, akkor az új elemet is bevesszük a listába.
Minden megtalált reprezentáns elemnek kiszámolhatjuk a konjugáltosztályának a méretét, ugyanis az a centralizátorának indexével egyenlő.
Akkor fejezzük be az algoritmust, amikor az osztályok méretének összege eléri a csoport rendjét.
Ennek az algoritmusnak a nagy hátránya, hogy ha egy nagy csoportban van egy kis elemszámú konjugáltosztály, akkor azt csak nagyon nehezen tudjuk megtalálni.

Másik véletlenszerű elemválasztáson alapuló algoritmust talált ki Mark Jerrum 1995-ben \cite{Jer95}.
Konstruáljunk egy $M$ Markov-láncot a következőképpen.
Az állapotok halmaza legyen $G$, a $P$ átmeneti valószínűségek mátrixának $g, h \in G$-hez tartozó eleme pedig
\begin{equation*}
p_{g,h} =
\begin{cases}
1/|C_G(g)| & \text{ha $h \in C_G(g)$}\\
0 & \text{különben}
\end{cases}
\end{equation*}
$G$ elemein nézzünk egy véletlen sétát $M$ szerint. Először választunk egy $x_0 \in G$ kezdőállapotot, majd miután $(x_0, \dots, x_k)$ már definiált, legyen
$x_{k+1}$ véletlen elem $C_G(x_k)$-ból.
Mivel minden $g,h \in G$-re $1 \in C_G(g)$ és $h \in C_G(1)$, ezért akárhonnan akárhova legfeljebb két lépéssel eljuthatunk, tehát $M$ irreducibilis.
Minden $g$-re $p_{g,g} = 1/|C_G(g)| > 0$, tehát minden állapot aperiodikus.
Így alkalmazhatjuk a Markov-láncok alaptételét, miszerint létezik (és csak egy létezik) stacionárius eloszlás ($v = (v_g \mid g \in G)$) és $\delta \in (0,1)$, amire
minden $k \ge 0$-ra és $g \in G$-re $|\Prob(x_k = g) - v_g| \le \delta^k$.
Jelöljük $r$-rel a konjugált osztályok számát, és legyen $v_g = C_G(g)/r|G|$.
Könnyen látható, hogy $\sum_{g\in G} v_g = 1$, valamint hogy $v = vP$, tehát ez a $v$ a stacionárius eloszlás.
Ha akármelyik konjugáltosztályra összeadjuk az abba tartozó $g$-kre $v_g$-ket, akkor $1/r$-et kapunk, tehát elég nagy $m$-re $x_m$ közel egyenlő valószínűséggel lehet az egyes konjugáltosztályokban.
Tehát az előző algoritmusunkat változtassuk annyival, hogy ne teljesen véletlenszerű csoportelemeket vegyünk mindig, hanem mindig az előző centralizátorából válasszunk csak, és így gyorsabban megtalálhatjuk
minden osztály reprezentáns elemét.

Vannak teljesen más elven működő további algoritmusok is erre a célra, de azokat nem részletezzük.