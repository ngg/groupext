\subsection{Black-box csoportok}
\label{subsec:blackbox}
A black-box csoportok eredeti ötlete és eredeti definíciója magyaroktól származik,
Babai László és Szemerédi Endre publikálta \cite{BS84}-ben.
Ma a legtöbb helyen inkább egy kicsit általánosabb definíciót használnak, amivel kevésbé körülményes az
egyes csoportok tényleges leírása. Az itt leírt definíciót \cite{Ser03}-ból vettem.

A black-box csoport egy olyan csoport,
aminek elemeit a véges $\Sigma$ ábécé feletti legfeljebb $N$ hosszú szavaival azonosítjuk.
Nincs megkövetelve, hogy egy elemnek csak egy szó felelhessen meg, se az hogy minden szó hozzátartozzon
egy elemhez.
A csoportműveleteket egy orákulum végzi.
Ha adott két szó, amik a $g, h\in G$ elemeket reprezentálják,
akkor meg tudjuk állapítani, hogy $g=1$ igaz-e,
valamint ki tudjuk számolni a $g^{-1}$-hez és a $gh$-hoz tartozó szavakat.
Általában a csoport megadása egy generátorrendszer segítségével történik, az elemeihez tartozó szavak megadásával.

Előfordulhat, hogy az orákulum egy $\overline{G} \ge G$ nagyobb csoport elemeihez tartozó szavakat fogad el,
ilyenkor csak az tesszük fel, hogy azt tudjuk megállapítani, hogy a szó $\overline{G}$-beli-e,
azt nem, hogy $G$ vagy $\overline{G}\setminus G$ eleme.

Erre példa egy véges $F$ test feletti $G$ $n\times n$-es mátrixcsoport.
$\Sigma=F$ adja az ábécét, a szavak $n^2$ hosszúak,
valamint $\overline{G}$ az összes invertálható $F$ feletti $n\times n$-es mátrix csoportja.
Természetesen a permutációcsoportok, illetve a policiklikus csoportok is könnyen leírhatók black-box csoportokként.

A black-box csoportokra vonatkozó algoritmusoknak az egyik leggyakoribb csoportja az általunk később is sokat használt
orbit-számoló algoritmusok, amiket az \ref{subsubsec:orbit}. alfejezetben részletezünk.
Randomizált algoritmusok esetében nagyon fontos, hogy tudjunk jó közelítéssel uniform véletlen csoportelemet választani,
erről szól az \ref{subsubsec:veletlen}. alfejezet.

\subsubsection{Orbit-algoritmusok}
\label{subsubsec:orbit}
Gyakori eset, hogy $G$ hatását nézzük egy $\Omega$ halmazon, és ott egy $\alpha\in\Omega$ elem orbitját,
vagyis $\alpha^G = \{ \alpha^g \mid g \in G \}$-t keressük. Ilyen előfordulhat, ha $G \le \Sym(\Omega)$,
de például az is ide tartozik, ha $g \in G$ konjugáltosztályát szeretnénk meghatározni, ugyanis választhatjuk
$\Omega=G$-t és hatásnak a konjugálást tekinthetjük.

Feltesszük, hogy adott $\beta\in\Omega$ és $g\in G$-re meg tudjuk határozni a $\beta^g$ képet, valamint $\Omega$-beli
elemeket össze tudunk hasonlítani egymással. Lehetséges, hogy $\Omega$ nagyon nagy, így nem tesszük fel, hogy fel tudjuk sorolni az elemeit.

Legyen $G=\langle S\rangle$. $\alpha^G$ az a legszűkebb részhalmaza $\Omega$-nak, ami tartalmazza $\alpha$-t, valamint zárt $S$ elemeinek hatására nézve.
Vegyük azt az irányított gráfot, aminek csúcsai $\Omega$ elemei és $\beta$-ból $\gamma$-ba akkor megy él, ha van olyan $S$-beli elem, ami $\beta$-t $\gamma$-ba viszi.
Ebben a gráfban kell megkeresnünk az $\alpha$-t tartalmazó összefüggő komponenst.
Mivel $G$ véges rendű, így $\beta^g=\gamma$ esetén $\exists l \gamma^{g^l}=\beta$, vagyis az összefüggő komponensek erősen osszefüggők.
A komponenst meg tudjuk találni szélességi kereséssel $O(|S| |\alpha^G|)$ idő alatt, $O(|\alpha^G|)$ memória használatával.

Most nézzünk egy általánosabb verziót.
Legyen adott $\Omega$-n egy algebrai struktúra, és $A \subseteq \Omega$-ra szeretnénk kiszámolni $\langle A^G \rangle$-t, vagyis
$\Omega$-nak azt a legszűkebb részhalmazát, ami tartalmazza $A$-t és zárt $G$ hatására és az $\Omega$-beli művelet(ek)re is.
Ha az előző konjugálós példát nézzük úgy, hogy $\Omega$-n is nézzük a csoportműveletet, akkor ezzel $A$ normál lezártját szeretnénk meghatározni.
Ha $G'=[G,G]$ kommutátor részcsoportot szeretnénk meghatározni,
akkor azt így tudjuk megtenni, hiszen $G'=\langle[s,t] \mid s,t \in S\rangle_N$, vagyis az $S$-beli elemek kommutátorainak normál lezártja.
Ezzel tudunk feloldhatóságot valamint nilpotenciát is ellenőrizni.

Mivel a végeredmény zárt $\Omega$-beli műveletekre, így célszerű az algoritmusnak csak egy generátorrendszert előállítani.
Feltesszük, hogy tudjuk ellenőrizni, hogy $\Omega$ egy eleme benne van-e pár $\Omega$-beli elem által generált struktúrában.
Az algoritmus az előzőnek egy egyszerű változtatásával kapható, egyrészt az elején a szélességi bejárást $A$ minden eleméből egyszerre kell indítani,
másrészt ha egy új csúcshoz érünk, akkor csak akkor kell felírnunk a generátorelemek listájába, ha az addigiak által generált struktúrában nincs benne.
Érdemes itt megjegyeznünk, hogy ha $\Omega$-n legalább egy csoportstruktúra van, akkor minden új generátorelem a csoport elemszámát legalább megduplázza,
vagyis legfeljebb $O(\log(|\langle A^G\rangle |))$ generátorelemet írunk fel. A számítási idő nagyrésze jellemzően arra a számolásra megy el, amikor egy elemről
megpróbáljuk eldönteni, hogy benne van-e a már felírt elemek által generált struktúrában. Vannak olyan randomizált algoritmusok is, amikhez nem szükséges feltennünk,
hogy ezt el tudjuk dönteni, ezekről \cite{Ser03}-ban olvashatunk.

\subsubsection{Véletlen csoportelem választása}
\label{subsubsec:veletlen}
\todo{alalfejezet: Véletlen csoportelem választása}
