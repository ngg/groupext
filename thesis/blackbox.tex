\subsection{Black-box csoportok}
\label{subsec:blackbox}
A black-box csoportok eredeti ötlete és eredeti definíciója magyaroktól származik,
Babai László és Szemerédi Endre publikálta \cite{BS84}-ben.
Ma a legtöbb helyen inkább egy kicsit általánosabb definíciót használnak, amivel kevésbé körülményes az
egyes csoportok tényleges leírása. Az itt leírt definíciót \cite{Ser03}-ból vettem.

A black-box csoport egy olyan csoport,
aminek elemeit a véges $\Sigma$ ábécé feletti legfeljebb $N$ hosszú szavaival adjuk meg.
Nincs megkövetelve, hogy egy elemnek csak egy szó felelhessen meg, se az hogy minden szó hozzátartozzon
egy elemhez.
A csoportműveleteket egy orákulum végzi.
Ha adott két szó, amik a $g, h\in G$ elemeket reprezentálják,
akkor meg tudjuk állapítani, hogy $g=1$ igaz-e,
valamint ki tudjuk számolni a $g^{-1}$-hez és a $gh$-hoz tartozó szavakat.

Előfordulhat, hogy az orákulum egy $\overline{G} \ge G$ nagyobb csoport elemeihez tartozó szavakat fogad el,
ilyenkor csak az tesszük fel, hogy azt tudjuk megállapítani, hogy a szó $\overline{G}$-beli-e,
azt nem, hogy $G$ vagy $\overline{G}\setminus G$ eleme.

Erre példa egy véges $F$ test feletti $G$ $n\times n$-es mátrixcsoport.
$\Sigma=F$ adja az ábécét, a szavak $n^2$ hosszúak,
valamint $\overline{G}$ az összes invertálható $F$ feletti $n\times n$-es mátrix csoportja.

Természetesen a permutációcsoportok, illetve a policiklikus csoportok is könnyen leírhatók black-box csoportokként.