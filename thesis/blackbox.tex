\subsection{Black-box csoportok}
\label{subsec:blackbox}
A black-box csoportok eredeti ötlete és eredeti definíciója magyaroktól származik,
Babai László és Szemerédi Endre publikálta \cite{BS84}-ben.
Ma a legtöbb helyen inkább egy kicsit általánosabb definíciót használnak, amivel kevésbé körülményes az
egyes csoportok tényleges leírása. Az itt leírt definíciót \cite{Ser03}-ból vettem.

A black-box csoport egy olyan csoport,
aminek elemeit a véges $\Sigma$ ábécé feletti legfeljebb $N$ hosszú szavaival azonosítjuk.
Nincs megkövetelve, hogy egy elemnek csak egy szó felelhessen meg, se az hogy minden szó hozzátartozzon
egy elemhez.
A csoportműveleteket egy orákulum végzi.
Ha adott két szó, amik a $g, h\in G$ elemeket reprezentálják,
akkor meg tudjuk állapítani, hogy $g=1$ igaz-e,
valamint ki tudjuk számolni a $g^{-1}$-hez és a $gh$-hoz tartozó szavakat.
Általában a csoport megadása egy generátorrendszer segítségével történik, az elemeihez tartozó szavak megadásával.

Előfordulhat, hogy az orákulum egy $\overline{G} \ge G$ nagyobb csoport elemeihez tartozó szavakat fogad el,
ilyenkor csak az tesszük fel, hogy azt tudjuk megállapítani, hogy a szó $\overline{G}$-beli-e,
azt nem, hogy $G$ vagy $\overline{G}\setminus G$ eleme.

Erre példa egy véges $F$ test feletti $G$ $n\times n$-es mátrixcsoport.
$\Sigma=F$ adja az ábécét, a szavak $n^2$ hosszúak,
valamint $\overline{G}$ az összes invertálható $F$ feletti $n\times n$-es mátrix csoportja.
Természetesen a permutációcsoportok, illetve a policiklikus csoportok is könnyen leírhatók black-box csoportokként.

A black-box csoportokra vonatkozó algoritmusoknak az egyik leggyakoribb csoportja az általunk később is sokat használt
orbit-számoló algoritmusok, amiket az \ref{subsubsec:orbit}. alfejezetben részletezünk.
Randomizált algoritmusok esetében nagyon fontos, hogy tudjunk jó közelítéssel uniform véletlen csoportelemet választani,
erről szól az \ref{subsubsec:veletlen}. alfejezet.

\subsubsection{Orbit-algoritmusok}
\label{subsubsec:orbit}
Gyakori eset, hogy $G$ hatását nézzük egy $\Omega$ halmazon, és ott egy $\alpha\in\Omega$ elem orbitját,
vagyis $\alpha^G = \{ \alpha^g | g \in G \}$-t keressük. Ilyen előfordulhat, ha $G \le \mathop{Sym}(\Omega)$,
de például az is ide tartozik, ha $g \in G$ konjugáltosztályát szeretnénk meghatározni, ugyanis választhatjuk
$\Omega=G$-t és hatásnak a konjugálást tekinthetjük.

Feltesszük, hogy adott $\beta\in\Omega$ és $g\in G$-re meg tudjuk határozni a $\beta^g$ képet, valamint $\Omega$-beli
elemeket össze tudunk hasonlítani egymással. Lehetséges, hogy $\Omega$ nagyon nagy, így nem tesszük fel, hogy fel tudjuk sorolni az elemeit.

Legyen $G=<S>$. $\alpha^G$ az a legszűkebb részhalmaza $\Omega$-nak, ami tartalmazza $\alpha$-t, valamint zárt $S$ elemeinek hatására nézve.
Vegyük azt az irányított gráfot, aminek csúcsai $\Omega$ elemei és $\beta$-ból $\gamma$-ba akkor megy él, ha van olyan $S$-beli elem, ami $\beta$-t $\gamma$-ba viszi.
Ebben a gráfban kell megkeresnünk az $\alpha$-t tartalmazó összefüggő komponenst.
Mivel $G$ véges rendű, így $\beta^g=\gamma$ esetén $\exists l \gamma^{g^l}=\beta$, vagyis az összefüggő komponensek erősen osszefüggők.
A komponenst meg tudjuk találni szélességi kereséssel $O(|S| |\alpha^G|)$ idő alatt, $O(|\alpha^G|)$ memóriával.

\subsubsection{Véletlen csoportelem választása}
\label{subsubsec:veletlen}