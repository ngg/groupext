\section{Csoportok megadása}
\label{sec:csoport}
Az algoritmikus csoportelméletben az egyik legelsőként felmerülő kérdés, hogy hogyan is adjuk meg a csoportot a számítógépnek.
Mi csak véges csoportokkal foglalkozunk, így az elsőre legkézenfekvőbbnek tűnő megoldás az, ha a csoportot a Cayley-táblázatával definiáljuk.
Ehhez beszámozzuk a csoportelemeket $1$-től $n$-ig ($n$-nel jelöljük mostantól a csoport rendjét),
és egy $n\times n$-es mátrixban megadjuk minden két elem szorzatának indexét.
Ennek a módszernek a nyílvánvaló hátránya a nagyon nagy memóriaigény,
hiszen például egy $100000$-rendű csoport megadásához így $10$ milliárd indexet kell felírnunk, azaz 40 GB memóriára lenne szükségünk.

Az igazából használt módszereknek egytől-egyig megvan az a hátrányuk, hogy mind kihasználják a csoportnak a speciális szerkezetét.
A három leggyakrabban használt módszer a csoportot, mint véges halmazon ható permutációcsoportként, gyűrű feletti mátrixcsoportként, illetve végesen prezentált csoportként adja meg.
Permutációcsoportok megadásával foglalkozunk az \ref{subsec:permutacio} alfejezetben.
Mátrixcsoportokkal nem foglalkozunk, az egy különálló szakdolgozat témáját képezhetné.
Végesen prezentált csoportok egy speciális típusáról, a policiklikus csoportokról lesz szó az \ref{subsec:policiklikus} alfejezetben.

\todo{bekezdés: fekete doboz csoportok}

\subsection{Permutációcsoportok}
\label{subsec:permutacio}
Minden véges csoport felírható permutációcsoportként, így a permutációcsoportok hatékony használata kiemelten fontos.
A ma is használt módszer Charles C. Sims-től származik 1970-ből (\cite{Sim70}), a hatékonyságán Donald E. Knuth ötlete javított sokat,
amit 1991-ben publikált (\cite{Knu91}).
Legyen $G \le S_n = \mathop{Sym}(\{1, 2, \dots, n\})$.

\subsubsection{Bázisok és erős generátorrendszerek}
\todo{alfejezet: bázisok és erős generátorrendszerek}
Definíció, miért jó, elemeket hogy lehet prezentálni, Schreier-Sims

\subsubsection{Alapvető algoritmusok}
\todo{alfejezet: alapvető algoritmusok}
Csoport rendje, csoporttagság ellenőrzése, ...

\subsubsection{Backtrack algoritmusok}
\todo{alalfejezet: backtrack algoritmusok}
Centralizátor, normalizátor, konjugáltosztályok, ...

\subsection{Policiklikus csoportok}
\label{subsec:policiklikus}
Policiklikus csoportok

\subsubsection{Policiklikus prezentáció}
PC-relációk, elemek prezentálása

\subsubsection{Algoritmusok}
Orbit-stabilizátor módszerek, homomorfizmusok, részcsoportok, centralizátor, konjugáltosztályok, ...

\clearpage
