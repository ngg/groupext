\subsection{Permutációcsoportok}
\label{subsec:permutacio}
Minden véges csoport felírható permutációcsoportként, így a permutációcsoportok hatékony használata kiemelten fontos.
A ma is használt módszer Charles C. Sims-től származik 1970-ből (\cite{Sim70}), a hatékonyságán Donald E. Knuth ötlete javított sokat, amit 1991-ben publikált (\cite{Knu91}).
Az \ref{subsubsec:permdef}. alfejezet az alapvető definíciókról szól, amik szükségesek Sims módszerének megértéséhez,
az \ref{subsubsec:permbasic}. alfejezet pár egyszerű ehhez a módszerhez kapcsolódó algoritmust tárgyal,
az \ref{subsubsec:permss}. alfejezet arról szól, hogy hogyan tudunk egy tetszőleges módon megadott permutációcsoportnak a hatékony megadását megkonstruálni,
míg az \ref{subsubsec:permbt}. rész az itt előforduló bonyolultabb, főként backtrack-jellegű algoritmusokat részletezi.
Ez a fejezet egy nagyon rövid kivonata lényegében Seress Ákos közel 300 oldalas könyvének (\cite{Ser03}), ami részletesebben tárgyal erről a témakörröl.

\subsubsection{Bázisok és erős generátorrendszerek}
\label{subsubsec:permdef}
Legyen $G \le S_n = \mathop{Sym}(\Omega)$.
A $B=(\beta_1,\beta_2,\dots,\beta_m)$ különböző $\Omega$-beli elemekből álló sorozatot $G$ bázisának hívjuk,
ha $G$-nek egyetlen olyan eleme van (mégpedig az egységelem), ami pontonként fixen tartja $B$-t, vagyis ha $G_B=\{1\}$.
Egy bázis mindig definiál egy
\begin{equation}
\label{eq:permlanc}
G = G^{[1]} \ge G^{[2]} \ge \dots \ge G^{[m]} \ge G^{[m+1]} = 1
\end{equation}
részcsoport-láncot, ahol $G^{[i]}=G_{(\beta_1,\beta_2,\dots,\beta_{i-1})}=G_{\beta_1}\cap G_{\beta_2}\cap\dots\cap G_{\beta_{i-1}}$, vagyis a $B$ első $i-1$ elemét fixen tartó csoportelemek részcsoportja.
A bázis irredundáns, ha $\forall 1\le i \le m$-re $G^{[i]} > G^{[i+1]}$, vagyis ha a lánc $m+1$ különböző részcsoportból áll.
Mostantól csak irredundáns bázisokkal foglalkozunk.
$G^{[i]}$-ben $G^{[i+1]}$ mellékosztályai a $\beta_i^{G^{[i]}}$ orbit elemeinek felel meg, így $2 \le \left| G^{[i]} : G^{[i+1]} \right| \le n$.
Ezt felhasználva
\begin{equation}
\label{eq:permmeret}
2^m \le |G|=\prod_{i=1}^m \left| G^{[i]} : G^{[i+1]} \right| \le n^m
\end{equation}
Logaritmust véve, majd átrendezve
\begin{equation}
\label{eq:permmeret2}
\frac{\log |G|}{\log n} \le m = |B| \le \log |G|
\end{equation}
Különböző irredundáns bázisok lehetnek különböző méretűek, de nem lehet egy bázis "túl nagy" emiatt.

$S \subseteq G$ részhalmazt $G$ erős generátorrendszerének hívjuk (a $B$ bázisra nézve), ha $\forall 1\le i \le m+1$-re $\langle S \cap G^{[i]} \rangle = G^{[i]}$.
Ha adott egy erős generátorrendszer, akkor az \ref{subsubsec:orbit}. fejezetben leírt algoritmussal a $\beta_i^{G^{[i]}}$ orbitok könnyen kiszámolhatóak.
Ezeket az orbitokat fundamentális orbitoknak hívjuk.
Ha a kiszámolásuk során az orbit minden pontjához felírjuk, hogy melyik $G$-beli elem viszi oda $\beta_i$-t, akkor megkapjuk az $R_i$ transzverzálist,
vagyis a $G^{[i+1]}$ mellékosztályainak reprezentáns elemeit.
Feltehetjük, hogy $1 \in R_i$, vagyis hogy az egységelemet írtuk fel olyan elemnek ami $\beta_i$-t önmagába viszi. Ezeknek a transzverzálisoknak a segítségével
$\forall g \in G$ kanonikus alakra hozható, azaz egyértelműen felírható ilyen elemek szorzataként.
Precízebben megfogalmazva $\forall g \in G$-nek pontosan egy olyan szorzatalakja létezik, amire $g = r_m r_{m-1} \dots r_1$, ahol $r_i \in R_i$.
Ez a szorzatalak algoritmikusan könnyen megtalálható a következőképpen: Adott $g \in G$-re először megkeressük azt az $r_1 \in R_1$-et, amire $\beta_1^g = \beta_1^{r_1}$.
Ezután $g_2 = g r_1^{-1}$-zel folytatjuk, megkeressük azt az $r_2 \in R_2$-t, amire $\beta_2^{g_2} = \beta_2^{r_2}$. $g_3 = g_2 r_2^{-1}$, és így tovább, folytatjuk amíg végig nem érünk.
$1 = g_{m+1} = g_{m} r_m^{-1}$ lesz az utolsó lépés, visszaszámolva $g = r_m r_{m-1} \dots r_1$-et kapjuk. Ezt az eljárást szitáló eljárásnak hívjuk.

A transzverzálisokat, ha explicit számolnánk ki és tárolnánk, akkor $\Theta(n^2)$ idő és memória kellene hozzá.
Ezt elkerülhetjük az úgy nevezett Schreier-fák használatával.
A Schreier-fa adatszerkezet egy $(S_i, T_i)$ párokból álló sorozat, minden $\beta_i$ bázisponthoz tartozik egy pár.
$T_i$ egy irányított cimkézett fa, aminek pontjai a $\beta_i^{G^{[i]}}$ fundamentális orbit elemeinek felelnek meg.
Minden él a $\beta_i$ gyökér fele mutat és meg van cimkézve $S_i \subseteq G^{[i]}$ egy elemével.
Ha $\gamma_1$-ből $\gamma_2$-be megy egy él $h$ cimkével, az azt jelenti, hogy $\gamma_1^h=\gamma_2$.
Így ha akármelyik $\gamma$-ból végigmegyünk éleken $\beta_i$-ig és a cimkéket összeszorozzuk, akkor megkapjuk hogy melyik
permutáció viszi $\gamma$-t $\beta_i$-be. Így az $(S_i, T_i)$ pár meghatározza a $G^{[i+1]}$ részcsoport mellékosztályainak 
reprezentáns elemeinek az inverzét $G^{[i]}$-ben. Azért az inverzekkel csináltuk, mert a szitáláshoz az inverzekre van szükségünk.
A gyakorlatban $S_i$ felírása után (ehhez $O(|S_i|n)$ memória szükséges) $T_i$-t el tudjuk tárolni egy $n$-hosszú $V_i$ tömbben.
$\gamma \in \Omega$-ra $V_i[\gamma]$-t akkor és csak akkor definiáljuk, ha $\gamma \in \beta_i^{G^{[i]}}$,
és ilyenkor $V_i[\gamma]$ egy $S_i$ egy elemére mutató pointert tartalmaz, ami megfelel a $\gamma$-ból kiinduló egyetlen él cimkéjének.
Emiatt a tárolási mód miatt Sims eredetileg Schreier-vektoroknak hívta az adatszerkezetet.

\subsubsection{Alapvető algoritmusok}
\label{subsubsec:permbasic}
\todo{alfejezet: alapvető algoritmusok}
Csoport rendje, csoporttagság ellenőrzése, ...

\subsubsection{Schreier-Sims algoritmus}
\label{subsubsec:permss}

Erős generátorrendszerek kiszámolása a Schreier-Sims algoritmussal történik, ami a következő két lemmán alapszik.
Ha $H \le G$ és $R$ egy jobb transzverzálisa $H$-nak $G$-ben, akkor minden $g \in G$-re jelöljük
$Hg\cap R$ egyetlen elemét $\overline{g}$-vel.
\begin{lemma}
\label{thm:sims1}
Legyen $H \le G = \langle S \rangle$ és legyen $R$ jobb transzverzálisa $H$-nak $G$-ben, amire $1\in R$.
Ilyenkor a
\begin{equation*}
T=\left\{ rs(\overline{rs})^{-1} \mid r\in R, s\in S \right\}
\end{equation*}
halmaz $H$-t generálja, vagyis $H=\langle T \rangle$.
\end{lemma}
\begin{proof}
Definíció szerint $T \subseteq H$, így elég belátni, hogy $H \le \langle T \rangle$.
Legyen $h \in H$ tetszőleges, felírhatjuk $h = s_1 s_2 \dots s_k$ alakban, ahol $s_i \in S$.
Sorba definiáljuk $1 \le i \le k$-ra $r_i$-t és $t_i$-t, úgy hogy $h = t_1 t_2 \dots t_i r_i s_{i+1} s_{i+2} \dots s_k$ igaz legyen $\forall i$-re.
Kezdőértéknek vegyük $r_0=1$-et, ezzel $h = r_0 s_1 s_2 \dots s_k$.
Ha $r_{i-1}$ már definiált, akkor legyen $t_i=r_{i-1} s_i (\overline{r_{i-1} s_i})^{-1} \in T$ és $r_i = \overline{r_{i-1} s_i} \in R$.
Ezekre indukció szerint $h = t_1 t_2 \dots t_i r_i s_{i+1} s_{i+2} \dots s_k$ teljesül.
Ha végigértünk, akkor $h = t_1 t_2 \dots t_k r_k$ alakot kapunk.
Mivel $h \in H$ és $t_1 t_2 \dots t_k \in \langle T \rangle \le H$, ezért $r_k \in H\cap R = \{1\}$.
Így $h \in \langle T \rangle$.
\end{proof}
\begin{lemma}
\label{thm:sims2}
Legyen $\{\beta_1, \beta_2, \dots, \beta_k\} \subseteq \Omega$ és $G \le \mathop{Sym}(\Omega)$. $1 \le i \le k+1$-re legyen $S_i \subseteq G^{[i]} = G_{(\beta_1, \beta_2, \dots, \beta_{i-1})}$,
amikre $\langle S_i \rangle \ge \langle S_{i+1} \rangle$ teljesül $1 \le i \le k$-ra.
Ha $\langle S_1 \rangle = G$, $S_k = \emptyset$ és $\langle S_i \rangle_{\beta_i} = \langle S_{i+1} \rangle$ is teljesül $1 \le i \le k$-ra, akkor $B = (\beta_1, \beta_2, \dots, \beta_k)$
$G$-nek bázisa, és $S = \bigcup_{i=1}^k S_i$ erős generátorrendszer $B$-re nézve.
\end{lemma}
\begin{proof}
Teljes indukciót alkalmazunk, az indukciós feltevésünk az, hogy a $G' = \langle S_2 \rangle = G_{\beta_1}$ csoportnak $S' = \bigcup_{i=2}^k S_i$ erős generátorrendszere a $B' = \{\beta_2, \beta_3, \dots, \beta_k\}$ bázisra nézva.
Ellenőriznünk kell, hogy $\forall 1\le i \le k+1$-re $\langle S \cap G^{[i]} \rangle = G^{[i]}$.
$G^{[1]} = G$, így $i = 1$-re triviális.
A lemma feltevéséből $G^{[2]} = G_{\beta_1} = \langle S_1 \rangle_{\beta_1} = \langle S_2 \rangle \le \langle S \cap G_{\beta_1} \rangle \le G_{\beta_1} = G^{[2]}$, vagyis $i = 2$-re is készen vagyunk.
$i > 2$-re az indukciós feltevés miatt $G^{[i]} \ge \langle S \cap G_{(\beta_1, \dots, \beta_{i-1})} \rangle \ge \langle S' \cap G'_{(\beta_2, \dots, \beta_{i-1})} \rangle = G'_{(\beta_2, \dots, \beta_{i-1})} = G_{(\beta_1, \dots, \beta_{i-1})} = G^{[i]}$,
tehát $i > 2$-re is készen vagyunk.
Mivel az indukciós feltevést csak $k \ge 2$ esetén használtuk, a bizonyítással készen vagyunk.
\end{proof}
\todo{schreier-sims algoritmus}

\subsubsection{Backtrack algoritmusok}
\label{subsubsec:permbt}
\todo{alalfejezet: backtrack algoritmusok}
Centralizátor, normalizátor, konjugáltosztályok, ...
