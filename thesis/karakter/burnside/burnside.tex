18;5~\subsection{Burnside-Dixon-Schneider algoritmus}
\label{subsec:bds}
Az első algoritmust karaktertáblázatok számítására Burnside írta le \cite{Bur11}-ben. Azóta is az ő módszere, és annak átalakított változatai
az egyetlen minden csoportra alkalmazható ismert algoritmus.
Ebben az alfejezetben ismertetjük az eredeti algoritmus ötleteit, illetve pár változtatást.
\todo{bekezdés: Mathematica kód}

\subsubsection{Burnside eredeti algoritmusa}
\label{subsubsec:bdseredeti}
Legyen $G$ tetszőleges véges csoport, amit multiplikatívnak tekintünk, egységelemét szokásosan $1$-gyel jelöljük.
Jelöljuk $G$ konjugáltosztályainak számát $r$-rel, az osztályokat $C_1, C_2, \dots, C_r$-rel, ezeknek egy-egy reprezentáns elemét $g_1, g_2, \dots, g_r$-rel.
Legyen $i=1, 2, \dots, r$-re $h_i=|C_i|=|G:C_G(C_i)|$, a konjugáltosztályok elemszáma.
Természetesen választhatjuk $g_1$-et $1$-nek, így $h_1=1$.

Feltételezzük, hogy az \ref{sec:csoport} fejezetben leírtak szerint $g_i$-k, $h_i$-k már ki vannak számolva.
Azt is fel kell tennünk, hogy minden $g\in G$-re gyorsan meg tudjuk állapítani, hogy melyik konjugáltosztályba tartozik.
Kisebb csoportok esetében, ha van elég memória, érdemes minden $g$-re előre kiszámolnunk ezt.

Jelöljük $G$ irreducibilis karaktereit $\chi^1, \chi^2, \dots, \chi^r$-rel, és a rövidség kedvéért legyen $\chi^i_j=\chi^i(g_j)$.
Legyen $d_i=\chi^i_1$, azaz $\chi^i$ foka.

$1 \le j, k, l \le r$-re legyen $c_{jkl}$ azon elempárok száma, ahol az egyik elem a $C_j$-beli, a másik elem $C_k$-beli és a szorzatuk $g_l$.
Ismert, hogy $c_{jkl}$ független $g_l$ megválasztásától, valamint hogy teljesül a következő egyenlőség $\forall 1 \le i, j, k \le r$-re:
\begin{align}
\label{eq:bdscjkl1}
\frac{h_j\chi^i_j}{d_i}\frac{h_k\chi^i_k}{d_i}=\sum_{l=1}^r c_{jkl}\frac{h_l\chi^i_l}{d_i}
\end{align}
Legyen $M_j$ az az $r\times r$-es mátrix, aminek $(k,l)$-edik eleme $c_{jkl}$.
Legyen $v^i=[h_1\chi^i_1/d_i, h_2\chi^i_2/d_i, \dots, h_r\chi^i_r/d_i]^T$.
Ha $i, j$-t lerögzítjük, akkor ezekkel a jelölésekkel a (\ref{eq:bdscjkl1}) egyenlet átírható a következő alakba:
\begin{align}
\label{eq:bdscjkl2}
\frac{h_j\chi^i_j}{d_i}v^i=M_j v^i
\end{align}
Vagyis $v^i$ jobboldali sajátvektora $M_j$-nek $\forall i, j$-re.
Van tehát $r$ darab közös sajátvektorunk minden $M_j$-hez, amik közül semelyik kettőhöz sem tartozhat minden $M_j$ esetén ugyanazok a sajátértékek,
így ezeket a sajátvektorokat $M_j$-kből konstansszorzó erejéig egyértelműen meghatározhatjuk lineáris algebrai ismereteink alapján.
A helyes konstansszorzókat könnyen megkaphatjuk, hiszen $v^i_1=h_1\chi^i_1/d_i=1 d_i/d_i=1$, így $M_j$-kből meg tudjuk állapítani $v^i_j$-ket.

Kérdés, hogy $v^i_j$-k ismeretében hogyan állapíthatjuk meg $\chi^i_j$-ket.
$h_j$-ket ismerjük, tehát elegendő $d_i$-k kiszámolása.
Szintén ismert, hogy a komplex-értékű osztályfüggvények terén értelmezhető egy természetes skalárszorzat a következőképpen:
\begin{align}
\label{eq:bdsscalar}
\langle \alpha, \beta \rangle = \frac{1}{|G|}\sum_{g\in G}\alpha(g)\overline{\beta(g)} = \frac{1}{|G|}\sum_{j=1}^r h_j \alpha(g_j) \overline{\beta(g_j)}
\end{align}
A karakterekre vonatkozó ortogonális relációk alapján $\langle \chi^i, \chi^j \rangle = \delta_{ij}$, ami alapján
\begin{align}
\label{eq:bdsdi1}
1 = \langle \chi^i, \chi^i \rangle = \frac{1}{|G|}\sum_{j=1}^r h_j |\chi^i_j|^2 = \frac{1}{|G|}\sum_{j=1}^r h_j \left|\frac{d_i v^i_j}{h_j}\right|^2 = 
\frac{d_i^2}{|G|}\sum_{j=1}^r \frac{|v^i_j|^2}{h_j}
\end{align}
Ezalapján $d_i$ kifejezhető, tehát sikerült $\chi^i_j$-ket kiszámolnunk.

\noindent
Összefoglalva az algoritmust:
\begin{enumerate}
\item Kiszámoljuk $r, C_i, g_i, h_i, M_i$-ket
\item Kiszámoljuk $M_i$-knek az $r$ darab közös sajátvektorát, majd ezekből $v^i_j$-ket
\item Kiszámoljuk $d_i$-ket, és végül ebből $\chi^i_j$-ket
\end{enumerate}

\subsubsection{Schneider első módosítása}
\label{subsubsec:bdssch1}
\todo{alalfejezet: Schneider 1}

\subsubsection{Dixon módosítása}
\label{subsubsec:bdsdixon}
\todo{alalfejezet: Dixon}

\subsubsection{Schneider második módosítása}
\label{subsubsec:bdssch2}
\todo{alalfejezet: Schneider 2}

\subsubsection{További módosítások}
\label{subsubsec:bdstovabbi}
\todo{alalfejezet: További módosítások}
