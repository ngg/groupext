\addcontentsline{toc}{section}{Bevezetés}
\section*{Bevezetés}
\label{sec:bevezetes}
Az algoritmikus csoportelmélet a matematikának egy olyan ága, amivel a XX. század eleje óta foglalkoznak,
de csak a '60-'70-es években kezdett el nagyon fejlődni.
A szakdolgozat kitűzött célja az, hogy véges csoportoknak számítógéppel tudjuk kiszámolni a karaktertáblázatát.
Ez egy elég összetett feladat ahhoz, hogy rengeteg más csoportelméleti algoritmust is magával húzzon,
ezekről fogunk beszélni.

A legtöbb csoportelméleti algoritmus a csoport számítógépes megadásától függ, ugyanis
a csoportot megadhatjuk permutációcsoportként vagy mátrixcsoportként; policiklikus reprezentációval vagy
véges generátorrendszerrel és relációkkal is.
Vannak azonban olyan algoritmusok is, amelyek nem függnek a megadás módszerétől,
itt a csoportra, mint egy fekete dobozra tekintünk, aminek nem ismerjük a szerkezetét.
Innen jön az ilyen csoportokra a black-box elnevezés.
Ezekre vonatkozó algoritmusokról szól az \ref{sec:blackbox}. fejezet.

A megadási módszerek közül a permutációcsoportokat választottam ki, hatékony használatukról és a hozzájuk kapcsolódó algoritmusokról szól a \ref{sec:permutacio}. fejezet.

Karaktertáblázatok kiszámolására jelenleg kettő, egymástól teljesen eltérő ötletre alapuló algoritmus ismert.
Ezek közül amiről mi beszélünk, az a Burnside-Dixon-Schneider algoritmus, ami egy lineáris algebrai megközelítést alkalmaz,
erről szól a \ref{sec:burnside}. fejezet.
A másik algoritmusról nem fogunk részletesen beszélni, itt szeretnék pár mondatot írni róla.
1986-ban Slattery \cite{Sla86} publikált egy eljárást, ami $p$-csoportokra meg tudja mondani, hogy hányadrendű irreducibilis reprezentációból hány darab van.
Conlon 1990-ben \cite{Con90} ezt továbbfejlesztette, és $p$-csoportokra ki tudta számolni a karaktertáblázatot ugyanazon ötlet alapján.
A minden csoportra alkalmazható algoritmus Unger-nek köszönhető 2006-ból \cite{Ung06}, az ötlet Brauer tételén alapszik, miszerint az általánosított karakterek gyűrűjét generálják
az úgynevezett elemi részcsoportok irreducibilis karaktereiből indukált karakterek.
Elemi részcsoportoknak azokat a részcsoportokat hívjuk, amelyek előállnak egy $p$-csoport és egy ciklikus csoport direktszorzataként,
ezeknek tehát Conlon algoritmusának segítségével ki tudjuk számolni az irreducibilis karaktereit.
Miután az algoritmus kiszámolta a gyűrű egy generátorrendszerét, LLL-redukció \cite{LLL82} segítségével megkeresi az 1-normájú karaktereket, ezek lesznek az irreducibilis karakterek.

A szakdolgozat függelékében megtalálható az algoritmusok nagyrészének implementációja Mathematica 8 környezetben.
Azért a Mathematica-ra esett a választásom, mivel az az egyik legnagyobb, legjobban megírt általános célú matematikai programcsomag, de az algebrai alkalmazások terén azonban nagyon elmaradott.
A 2010-ben kiadott 8-as verzió volt az első, amibe már valamennyi csoportelméleti funkciót beleraktak, azonban csak permutációcsoportokat tud kezelni, és azokra is csak nagyon kevés algoritmus van benne.
Az implementáció a meglevő függvényeket már felhasználja, így régebbi verziókkal nem kompatibilis.
Mivel publikussá szeretném tenni a programomat, ezért annak a dokumentációja, valamint a forráskódbeli megjegyzések angol nyelvűek.
A részletezett algoritmusokra példák a dokumentációban találhatóak.

A szakdolgozat előismeretnek tekinti a Matematika BSc négy félévnyi Algebra képzésének anyagát, valamint általában algoritmusokkal kapcsolatos jártasságot.
Akinek a téma felkeltette az érdeklődését az algoritmikus csoportelmélet iránt, az részletesebben olvashat erről \cite{HEO05}-ben,
valamint kifejezetten permutációcsoport-algoritmusokról \cite{Ser03}-ban.

Ezúton szeretnék köszönetet mondani témavezetőmnek, Pelikán Józsefnek, aki a négy félév Algebra előadásaival és gyakorlataival felkeltette érdeklődésemet a téma iránt,
valamint szakirodalom ajánlásával és értékes észrevételeivel nagyban segítette a szakdolgozat elkészítését.
