\addcontentsline{toc}{section}{Bevezetés}
\section*{Bevezetés}
\label{sec:bevezetes}
\todo{innen at kell irni}
Az algoritmikus csoportelméletben az egyik legelsőként felmerülő kérdés, hogy hogyan is adjuk meg
a csoportot a számítógépnek.
Mi csak véges csoportokkal foglalkozunk, így az elsőre legkézenfekvőbbnek tűnő megoldás az, ha a
csoportot a Cayley-táblázatával definiáljuk.
Ehhez beszámozzuk a csoportelemeket $1$-től $n$-ig ($n$-nel jelöljük mostantól a csoport rendjét),
és egy $n\times n$-es mátrixban megadjuk minden két elem szorzatának indexét.
Ennek a módszernek a nyílvánvaló hátránya a nagyon nagy memóriaigény,
hiszen például egy $100000$-rendű csoport megadásához így $10$ milliárd indexet kell felírnunk,
azaz (32-bites számítógéppel számolva) 40 GB memóriára lenne szükségünk.

Két nagy típusa van a valójában használt módszereknek.
Az első típus az úgynevezett black-box csoportok, itt a csoportokat mint egy fekete dobozt képzelünk el,
amiről nem tudjuk, hogy hogyan működik belül, nem teszünk fel semmit a szerkezetéről.
A black-box csoportokra vonatkozó algoritmusoknak megvan az a nagy előnye, hogy minden véges csoportra
alkalmazhatóak, de ez egyben a hátrányuk is, hiszen nem tudják kihasználni a csoportok speciális szerkezetét.
Részletesebben foglalkozunk ilyenekkel a \ref{subsec:blackbox}. alfejezetben.

A másik nagy típus az, amikor a csoportnak egy bizonyos szerkezetét kihasználva adjuk meg.
A három leggyakrabban használt ilyen módszer a csoportot, mint véges halmazon ható permutációcsoportként,
test feletti mátrixcsoportként, illetve végesen prezentált csoportként adja meg.
Permutációcsoportok megadásával foglalkozunk az \ref{subsec:permutacio}. alfejezetben.
Mátrixcsoportokkal nem foglalkozunk, az egy különálló szakdolgozat témáját képezhetné.
Végesen prezentált csoportok egy speciális típusáról, a policiklikus csoportokról lesz szó
az \ref{subsec:policiklikus}. alfejezetben.
\todo{idaig}


Téma, motiváció, felépítés, köszönetnyilvánítás
Mindig 2-es alapú logaritmust nézünk
\todo{fejezet: Bevezetés}
