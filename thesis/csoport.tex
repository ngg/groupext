\section{Csoportok megadása}
\label{sec:csoport}
Az algoritmikus csoportelméletben az egyik legelsőként felmerülő kérdés, hogy hogyan is adjuk meg
a csoportot a számítógépnek.
Mi csak véges csoportokkal foglalkozunk, így az elsőre legkézenfekvőbbnek tűnő megoldás az, ha a
csoportot a Cayley-táblázatával definiáljuk.
Ehhez beszámozzuk a csoportelemeket $1$-től $n$-ig ($n$-nel jelöljük mostantól a csoport rendjét),
és egy $n\times n$-es mátrixban megadjuk minden két elem szorzatának indexét.
Ennek a módszernek a nyílvánvaló hátránya a nagyon nagy memóriaigény,
hiszen például egy $100000$-rendű csoport megadásához így $10$ milliárd indexet kell felírnunk,
azaz (32-bites számítógéppel számolva) 40 GB memóriára lenne szükségünk.

Két nagy típusa van a valójában használt módszereknek.
Az első típus az úgynevezett black-box csoportok, itt a csoportokat mint egy fekete dobozt képzelünk el,
amiről nem tudjuk, hogy hogyan működik belül, nem teszünk fel semmit a szerkezetéről.
A black-box csoportokra vonatkozó algoritmusoknak megvan az a nagy előnye, hogy minden véges csoportra
alkalmazhatóak, de ez egyben a hátrányuk is, hiszen nem tudják kihasználni a csoportok speciális szerkezetét.
Részletesebben foglalkozunk ilyenekkel a \ref{subsec:blackbox}. alfejezetben.

A másik nagy típus az, amikor a csoportnak egy bizonyos szerkezetét kihasználva adjuk meg.
A három leggyakrabban használt ilyen módszer a csoportot, mint véges halmazon ható permutációcsoportként,
test feletti mátrixcsoportként, illetve végesen prezentált csoportként adja meg.
Permutációcsoportok megadásával foglalkozunk az \ref{subsec:permutacio}. alfejezetben.
Mátrixcsoportokkal nem foglalkozunk, az egy különálló szakdolgozat témáját képezhetné.
Végesen prezentált csoportok egy speciális típusáról, a policiklikus csoportokról lesz szó
az \ref{subsec:policiklikus}. alfejezetben.

\subsection{Black-box csoportok}
\label{subsec:blackbox}
A black-box csoportok eredeti ötlete és eredeti definíciója magyaroktól származik,
Babai László és Szemerédi Endre publikálta \cite{BS84}-ben.
Ma a legtöbb helyen inkább egy kicsit általánosabb definíciót használnak, amivel kevésbé körülményes az
egyes csoportok tényleges leírása. Az itt leírt definíciót \cite{Ser03}-ból vettem.

A black-box csoport egy olyan csoport,
aminek elemeit a véges $\Sigma$ ábécé feletti legfeljebb $N$ hosszú szavaival adjuk meg.
Nincs megkövetelve, hogy egy elemnek csak egy szó felelhessen meg, se az hogy minden szó hozzátartozzon
egy elemhez.
A csoportműveleteket egy orákulum végzi.
Ha adott két szó, amik a $g, h\in G$ elemeket reprezentálják,
akkor meg tudjuk állapítani, hogy $g=1$ igaz-e,
valamint ki tudjuk számolni a $g^{-1}$-hez és a $gh$-hoz tartozó szavakat.

Előfordulhat, hogy az orákulum egy $\overline{G} \ge G$ nagyobb csoport elemeihez tartozó szavakat fogad el,
ilyenkor csak az tesszük fel, hogy azt tudjuk megállapítani, hogy a szó $\overline{G}$-beli-e,
azt nem, hogy $G$ vagy $\overline{G}\setminus G$ eleme.

Erre példa egy véges $F$ test feletti $G$ $n\times n$-es mátrixcsoport.
$\Sigma=F$ adja az ábécét, a szavak $n^2$ hosszúak,
valamint $\overline{G}$ az összes invertálható $F$ feletti $n\times n$-es mátrix csoportja.

Természetesen a permutációcsoportok, illetve a policiklikus csoportok is könnyen leírhatók black-box csoportokként.
\section{Permutációcsoportok}
\label{sec:permutacio}
Minden véges csoport felírható permutációcsoportként, így a permutációcsoportok hatékony használata kiemelten fontos.
A ma is használt módszer Charles C. Sims-től származik 1970-ből \cite{Sim70}.
Az \ref{subsec:permdef}. alfejezet az alapvető definíciókról szól, amik szükségesek Sims módszerének megértéséhez,
az \ref{subsec:permss}. alfejezet arról szól, hogy hogyan tudjuk egy tetszőleges módon megadott permutációcsoportnak a hatékony megadását megkonstruálni,
míg az \ref{subsec:permbt}. rész az itt előforduló főként backtrack-jellegű algoritmusokat részletezi.
Ez a fejezet egy nagyon rövid kivonata lényegében Seress Ákos közel 300 oldalas könyvének \cite{Ser03}, ami részletesebben tárgyal erről a témakörröl.

\subsection{Bázisok és erős generátorrendszerek}
\label{subsec:permdef}
Legyen $G \le \Sym(\Omega)$.
A $B=(\beta_1,\dots,\beta_m)$ különböző $\Omega$-beli elemekből álló sorozatot $G$ bázisának hívjuk,
ha $G$-nek egyetlen olyan eleme van (mégpedig az egységelem), ami pontonként fixen tartja $B$-t, vagyis ha $G_B=\{1\}$.
Egy bázis mindig definiál egy
\begin{equation}
\label{eq:permlanc}
G = G^{[1]} \ge G^{[2]} \ge \dots \ge G^{[m]} \ge G^{[m+1]} = \{1\}
\end{equation}
részcsoport-láncot, ahol $G^{[i]}=G_{(\beta_1,\dots,\beta_{i-1})}=G_{\beta_1}\cap\dots\cap G_{\beta_{i-1}}$, vagyis a $B$ első $i-1$ elemét fixen tartó csoportelemek részcsoportja.
A bázis irredundáns, ha minden $1\le i \le m$-re $G^{[i]} > G^{[i+1]}$, vagyis ha a lánc $m+1$ különböző részcsoportból áll.
Mostantól csak irredundáns bázisokkal foglalkozunk.
$G^{[i]}$-ben $G^{[i+1]}$ mellékosztályai a $\beta_i^{G^{[i]}}$ orbit elemeinek felel meg, így $2 \le \left| G^{[i]} : G^{[i+1]} \right| \le |\Omega|$.
Ezt felhasználva
\begin{equation}
\label{eq:permmeret}
2^m \le |G|=\prod_{i=1}^m \left| G^{[i]} : G^{[i+1]} \right| \le |\Omega|^m
\end{equation}
Logaritmust véve (mint más algoritmusokkal kapcsolatos leírások esetén is, mi is mindig $2$-es alapú logaritmust használunk), majd átrendezve
\begin{equation}
\label{eq:permmeret2}
\frac{\log |G|}{\log |\Omega|} \le m = |B| \le \log |G|
\end{equation}
Különböző irredundáns bázisok lehetnek különböző méretűek, de nem lehet egy bázis "túl nagy" emiatt.

Az $S \subseteq G$ részhalmazt $G$ erős generátorrendszerének hívjuk (a $B$ bázisra nézve), ha minden $1\le i \le m+1$-re $\langle S \cap G^{[i]} \rangle = G^{[i]}$.
Ha adott egy erős generátorrendszer, akkor az \ref{subsec:orbit}. fejezetben leírt algoritmussal a $\beta_i^{G^{[i]}}$ orbitok könnyen kiszámolhatóak.
Ezeket az orbitokat fundamentális orbitoknak hívjuk.
Ha a kiszámolásuk során az orbit minden pontjához felírjuk, hogy melyik $G^{[i]}$-beli elem viszi oda $\beta_i$-t, akkor megkapjuk az $R_i$ transzverzálist,
vagyis $G^{[i+1]}$ mellékosztályainak reprezentáns elemeit.
Ha az $R_i$ transzverzálist explicit számolnánk ki és tárolnánk, akkor $\Theta(|\Omega|^2)$ idő és memória kellene hozzá.
Ezt elkerülhetjük az úgy nevezett Schreier-fák használatával.
A Schreier-fa adatszerkezet egy $T_i$ sorozat, minden $\beta_i$ bázisponthoz tartozik egy.
$T_i$ egy irányított cimkézett fa, aminek pontjai a $\beta_i^{G^{[i]}}$ fundamentális orbit elemeinek felelnek meg.
Minden él a $\beta_i$ gyökér fele mutat és meg van cimkézve $S$ egy elemével.
Ha $\gamma_1$-ből $\gamma_2$-be megy egy él $h$ cimkével, az azt jelenti, hogy $\gamma_1^h=\gamma_2$.
Így ha akármelyik $\gamma$-ból végigmegyünk éleken $\beta_i$-ig és a cimkéket összeszorozzuk, akkor megkapjuk hogy melyik
permutáció viszi $\gamma$-t $\beta_i$-be. Így $T_i$ meghatározza a $G^{[i+1]}$ részcsoport mellékosztályainak 
reprezentáns elemeinek az inverzét $G^{[i]}$-ben. Azért az inverzekkel csináltuk, mert a lentebb leírt, úgynevezett szitáló eljáráshoz az inverzekre lesz szükségünk.
A gyakorlatban $S$ felírása után (ehhez $O(|S||\Omega|)$ memória szükséges) $T_i$-t el tudjuk tárolni egy $|\Omega|$-hosszú $V_i$ tömbben.
$\gamma \in \Omega$-ra $V_i[\gamma]$-t akkor és csak akkor definiáljuk, ha $\gamma \in \beta_i^{G^{[i]}}$,
és ilyenkor $V_i[\gamma]$ azt tartalmazza, hogy $S$ hányadik eleme van a $\gamma$-ból kiinduló egyetlen él cimkéjén.
Emiatt a tárolási mód miatt Sims eredetileg Schreier-vektoroknak hívta az adatszerkezetet, és ennek használatával $S$ és az összes transzverzális összesen $\Theta((|S|+m)|\Omega|)$ memóriában eltárolható.

Feltehetjük, hogy $1 \in R_i$, vagyis hogy az egységelemet írtuk fel olyan elemnek ami $\beta_i$-t önmagába viszi.
Ezeknek a transzverzálisoknak a segítségével minden $g \in G$ kanonikus alakra hozható, azaz egyértelműen felírható ilyen elemek szorzataként.
Precízebben megfogalmazva minden $g \in G$-nek pontosan egy olyan szorzatalakja létezik, amire $g = r_m r_{m-1} \dots r_1$, ahol $r_i \in R_i$.
Ez a szorzatalak algoritmikusan könnyen megtalálható a következőképpen: Adott $g \in G$-re először megkeressük azt az $r_1 \in R_1$-et, amire $\beta_1^g = \beta_1^{r_1}$.
Ezután $g_2 = g r_1^{-1}$-zel folytatjuk, megkeressük azt az $r_2 \in R_2$-t, amire $\beta_2^{g_2} = \beta_2^{r_2}$. $g_3 = g_2 r_2^{-1}$, és így tovább, folytatjuk amíg végig nem érünk.
$1 = g_{m+1} = g_{m} r_m^{-1}$ lesz az utolsó lépés, visszaszámolva $g = r_m r_{m-1} \dots r_1$-et kapjuk. Ezt az eljárást szitáló eljárásnak hívjuk.

A szitálás alkalmas egyben arra is, hogy egy $h \in \Sym(\Omega)$-beli elemről eldöntsük, hogy $G$-ben van-e.
Ha megpróbáljuk $h$-ra alkalmazni az eljárást és sikerrel járunk, akkor $h\in G$.
Két rész van az algoritmusban ahol elakadhatunk.
Lehet, hogy valamely $i \le m$-re $h_i = h r_1^{-1} r_2^{-1} \dots r_{i-1}^{-1}$-re nem találunk megfelelő $r_i$-t, mert $\beta_i^{h_i} \notin \beta_i^{G^{[i]}}$.
Lehet az is, hogy a végén $h_{m+1} \neq 1$.
Mindkét ilyen esetben nyilvánvalóan $h \notin G$, ilyenkor az utoljára kiszámolt $h_i$-t vagy $h_{m+1}$-et maradéknak hívjuk.

$G$ rendjét is könnyen kiszámolhatjuk egy erős generátorrendszer ismeretében, ugyanis $|G| = |\beta_1^{G^{[1]}}| |\beta_2^{G^{[2]}}| \dots |\beta_m^{G^{[m]}}| = |R_1| |R_2| \dots |R_m|$.

\subsection{Schreier-Sims algoritmus}
\label{subsec:permss}
Ha megadnak nekünk egy $G \le \Sym(\Omega)$ permutációcsoportot valahogyan (black-box csoportnak is tekinthetjük akár),
akkor ki kell tudnunk számolni egy bázisát és egy erős generátorrendszert, ahhoz hogy hatékonyan tudjunk vele dolgozni.
Ez a Schreier-Sims algoritmussal történik, ami a következő két lemmán alapszik.
\begin{lemma}[Schreier]
\label{thm:sims1}
Legyen $H \le G = \langle S \rangle$ és legyen $R$ jobb transzverzálisa $H$-nak $G$-ben, amire $1\in R$.
Jelöljük $g \in G$-re $Hg\cap R$ egyetlen elemét $\overline{g}$-vel.
Ilyenkor a
\begin{equation*}
T=\left\{ rs(\overline{rs})^{-1} \mid r\in R, s\in S \right\}
\end{equation*}
halmaz $H$-t generálja, vagyis $H=\langle T \rangle$. A $T$ halmaz elemeit $H$ Schreier-generátorainak hívjuk.
\end{lemma}
\begin{proof}
Definíció szerint $T \subseteq H$, így elég belátni, hogy $H \le \langle T \rangle$.
Legyen $h \in H$ tetszőleges, felírhatjuk $h = s_1 s_2 \dots s_k$ alakban, ahol $s_i \in S$.
Sorban definiáljuk $1 \le i \le k$-ra $r_i$-t és $t_i$-t, úgy hogy $h = t_1 t_2 \dots t_i r_i s_{i+1} s_{i+2} \dots s_k$ igaz legyen $\forall i$-re.
Kezdőértéknek vegyük $r_0=1$-et, ezzel $h = r_0 s_1 s_2 \dots s_k$.
Ha $r_{i-1}$ már definiált, akkor legyen $t_i=r_{i-1} s_i (\overline{r_{i-1} s_i})^{-1} \in T$ és $r_i = \overline{r_{i-1} s_i} \in R$.
Ezekre indukció szerint $h = t_1 t_2 \dots t_i r_i s_{i+1} s_{i+2} \dots s_k$ teljesül.
Ha végigértünk, akkor $h = t_1 t_2 \dots t_k r_k$ alakot kapunk.
Mivel $h \in H$ és $t_1 t_2 \dots t_k \in \langle T \rangle \le H$, ezért $r_k \in H\cap R = \{1\}$.
Így $h \in \langle T \rangle$.
\end{proof}
\begin{lemma}
\label{thm:sims2}
Legyen $\{\beta_1, \dots, \beta_m\} \subseteq \Omega$ és $G \le \Sym(\Omega)$.
Legyen $S_i \subseteq G^{[i]} = G_{(\beta_1, \dots, \beta_{i-1})}$ minden $1 \le i \le m$-re.
A rövidség kedvéért vezessük be az $S_{m+1} = \emptyset$ jelölést.
Ha $\langle S_1 \rangle = G$ és $\langle S_i \rangle_{\beta_i} = \langle S_{i+1} \rangle$ teljesül $1 \le i \le m$-re, akkor $B = (\beta_1, \dots, \beta_m)$
$G$-nek bázisa, és $S = \bigcup_{i=1}^m S_i$ erős generátorrendszer $B$-re nézve.
\end{lemma}
\begin{proof}
Teljes indukciót alkalmazunk, az indukciós feltevésünk az, hogy a $G^{*} = \langle S_2 \rangle = G_{\beta_1}$ csoportnak $S^{*} = \bigcup_{i=2}^m S_i$ erős generátorrendszere a $B^{*} = (\beta_2, \dots, \beta_m)$ bázisra nézva.
Ellenőriznünk kell, hogy minden $1\le i \le m+1$-re $\langle S \cap G^{[i]} \rangle = G^{[i]}$.
$G^{[1]} = G$, így $i = 1$-re triviális.
A lemma feltevéséből $G^{[2]} = G_{\beta_1} = \langle S_1 \rangle_{\beta_1} = \langle S_2 \rangle \le \langle S \cap G^{[2]} \rangle \le G_{\beta_1} = G^{[2]}$, vagyis $i = 2$-re is készen vagyunk.
$i > 2$-re az indukciós feltevés miatt $G^{[i]} \ge \langle S \cap G_{(\beta_1, \dots, \beta_{i-1})} \rangle \ge \langle S^{*} \cap G^{*}_{(\beta_2, \dots, \beta_{i-1})} \rangle = G^{*}_{(\beta_2, \dots, \beta_{i-1})} = G_{(\beta_1, \dots, \beta_{i-1})} = G^{[i]}$,
tehát $i > 2$-re is készen vagyunk.
Mivel az indukciós feltevést csak $m \ge 2$ esetén használtuk, a bizonyítással készen vagyunk.
\end{proof}
Ha adott $G = \langle T \rangle$, akkor annak egy erős generátorrendszerét a következőképpen kaphatjuk.
Az algoritmus futása közben nyilvántartunk egy $B = (\beta_1, \dots, \beta_m)$ listát egy irredundáns bázis már ismert elemeiről,
és minden $1 \le i \le m$-re egy $S_i \subseteq G_{(\beta_1, \dots, \beta_{i-1})}$ halmazt, amikre a \ref{thm:sims2}. lemmát szeretnénk majd alkalmazni.
Az algoritmus alatt $S_i$-ket fogjuk növelni, illetve új báziselemeket fogunk $B$-hez adni (ezáltal $m$-et is növelve), így ha eleinte $\langle S_1 \rangle = G$ teljesül,
akkor abban a helyzetben vagyunk készen, amikor minden $1 \le i \le m$-re $\langle S_i \rangle_{\beta_i} = \langle S_{i+1} \rangle$ teljesül.
Végig fent fogjuk tartani, hogy $\langle S_i \rangle \ge \langle S_{i+1} \rangle$ is fennáll minden $i$-re.
Kezdetben legyen $B = (\beta_1)$, ahol $\beta_1 \in \Omega$ olyan hogy $T$-nek legalább egy eleme mozgatja, és legyen $S_1 = T$, ezzel garantálva, hogy $\langle S_1 \rangle = G$ teljesüljön.
Azt mondjuk, hogy az algoritmus az $l$. szinten tart, ha minden $l < i \le m$-re teljesül $\langle S_i \rangle_{\beta_i} = \langle S_{i+1} \rangle$.
Kezdetben az első szintről indulunk, az algoritmus futása akkor ér véget, amikor a nulladik szintre jutunk.

Amikor az $l$. szinten vagyunk, a következő történik:
Megnézzük, hogy $\langle S_l \rangle_{\beta_l} = \langle S_{l+1} \rangle$ teljesül-e.
Mivel $\langle S_l \rangle \ge \langle S_{l+1} \rangle$, valamint $S_l \subseteq G_{(\beta_1, \dots, \beta_{l-1})}$, ezért elegendő a $\langle S_l \rangle_{\beta_l} \le \langle S_{l+1} \rangle$ irányú tartalmazást ellenőrizni.
A \ref{thm:sims1}. lemma alapján $\langle S_l \rangle_{\beta_l} = \langle rs(\overline{rs})^{-1} \mid r\in R_l, s\in S_l \rangle$, ahol $R_l$ az $\langle S_l \rangle_{\beta_l}$ részcsoport transzverzálisa $\langle S_l \rangle$-ban.
Meg kell néznünk, hogy az összes Schreier-generátor benne van-e $\langle S_{l+1} \rangle$-ben.
Ezt az előző részben leírt szitálással tudjuk ellenőrizni, mivel a \ref{thm:sims2}. lemma szerint $\langle S_{l+1} \rangle$-nek már ismerjük egy erős generátorrendszerét.
Ha mind benne van, vagyis ha $\langle S_l \rangle_{\beta_l} = \langle S_{l+1} \rangle$ teljesül, akkor az $l-1$. szintre lépünk.
Ha nincs benne mind, vagyis $\langle S_l \rangle_{\beta_l} > \langle S_{l+1} \rangle$, akkor a szitálás során, amit kaptunk maradékot arra a Schreier-generátorra, ami nincs benne $\langle S_{l+1} \rangle$-ben,
azt vegyük hozzá az $S_{l+1}$ halmazhoz.
Ettől a hozzávételtől $\langle S_1 \rangle \ge \langle S_2 \rangle \ge \dots \ge \langle S_{m+1} \rangle$ továbbra is fennáll, valamint $S_{l+1} \subseteq G_{(\beta_1, \dots, \beta_{l})}$ is igaz marad.
$l = m$ esetén $B$-hez vegyünk hozzá egy olyan $\Omega$-beli elemet, amit a maradék nem hagy helyben.
Ezután az algoritmus az $l+1$. szinten folytatódik.

Az algoritmus mindenképpen véges, hiszen $\Omega$ véges, így csak véges sokszor tudtuk $B$-t növelni.
Ha a transzverzálisok számolásakor explicit leírunk minden elemet, akkor $O(|\Omega|^2 \log^3|G| + |T||\Omega|^2 \log|G|)$ a futásidő, és $O(|\Omega|^2 \log|G| + |T||\Omega|)$ a memóriaigény.
Ha Schreier-fákkal számoltunk, akkor $O(|\Omega|^3 \log^3|G| + |T||\Omega|^3 \log|G|)$ a futásidő, és $O(|\Omega| \log^2|G| + |T||\Omega|)$ a memóriaigény.
Ezeknek a bizonyítása megtalálható \cite{Ser03}-ban, valamint \cite{Mur03}-ban olvashatjuk az algoritmusnak rengeteg változatát részletesen elemezve, hogy mikor melyik a legalkalmasabb.

Sokszor elő fog fordulni, hogy az erős generátorrendszert nem akármilyen bázishoz szeretnénk megkapni.
Ilyenkor kétféle lehetőségünk van.
Ha megadnak nekünk egy $(\Omega, \prec)$ rendezést, amilyen sorrendben szeretnék a báziselemeket megkapni, akkor amikor az algoritmus új báziselemet választ, akkor mindig a lehető legelsőt választjuk,
illetve megnézzük a többi Schreier-generátort is, hogy azoknak a maradékával be tudunk-e venni előbbi báziselemet.
Másik lehetőség az úgynevezett bázisváltás, hogy ha már végigszámoltuk az algoritmust egy tetszőleges bázisban, akkor annak az eredményét felhasználva gyorsabban is ki lehet számolni az új erős generátorrendszert.
A módszer azon alapszik, hogy ha $(\beta_1, \dots, \beta_m)$ bázisra nézve már ismert az erős generátorrendszer és az $R_j$ transzverzálisok, akkor valamilyen $i$-re ki tudjuk számolni a $(\beta_1, \dots, \beta_{i-1}, \beta_{i+1}, \beta_i, \beta_{i+2}, \dots, \beta_m)$
bázishoz tartozót is, vagyis két egymásutáni báziselemet fel tudunk cserélni.
Ezzel a művelettel akármilyen más bázisba is eljuthatunk, hiszen új báziselemet is fel tudunk venni a listába olyan helyen, hogy az redundáns legyen.
Ez a módszer méghatásosabb, ha nem az egész bázist adták meg, hanem csak az első pár báziselemet.
Erre példa, ha egy $G_{\omega}$ stabilizátort szeretnénk kiszámolni, ilyenkor csak az első báziselem fontos.
Nézzük meg hogyan cserélhetünk ki két báziselemet.

Legyen $B = (\beta_1, \dots, \beta_m)$, $S$ az ehhez tartozó erős generátorrendszer, $R_j$-k pedig a transzverzálisak.
$\overline{B} = (\beta_1, \dots, \beta_{i-1}, \beta_{i+1}, \beta_i, \beta_{i+2}, \dots, \beta_m)$-ra ki szeretnénk számolni a hozzátartozó $\overline{S}$ erős generátorrendszert, valamint az új $\overline{R}_j$ transzverzálisokat.
Minden $1 \le j < i$-re, valamint $i+1 < j \le m$-re $\overline{R}_j = R_j$, tehát azokkal készen vagyunk.
Mivel $\overline{G}^{[i]} = G^{[i]}$, az új $i$-edik fundamentális orbit $\beta_{i+1}^{\langle S \cap G^{[i]} \rangle}$, ezért az $\overline{R}_i$ transzverzálist könnyen kiszámolhatjuk.
A számolás neheze $\overline{R}_{i+1}$-nél jön.
Legyen $\overline{S} = S$ eleinte, és kezdetben tekintsük az új $i+1$-edik fundamentális orbitot $\overline{\Delta}_{i+1} = \{\beta_i\}$-nek.
Ezekután tekinthetjük az $\overline{R}_i$ és $\overline{S}\cap \overline{G}^{[i]}$ által alkotott Schreier-generátorokat, amik $\overline{G}^{[i+1]} = G_{(\beta_1,\dots,\beta_{i-1},\beta_{i+1})}$ generátorrendszerét adják.
Ha ezek közül valamelyik $\beta_i$-t kiviszi $\overline{\Delta}_{i+1}$-ből, akkor azt az elemet $\overline{S}$-hoz adjuk hozzá és a fundamentális orbitot számoljuk újra:
$\overline{\Delta}_{i+1} = \beta_i^{\langle \overline{S} \cap \overline{G}^{[i+1]} \rangle}$.
Mivel $|\overline{\Delta}_{i+1}| = |\overline{R}_{i+1}| = |R_i| |R_{i+1}| / |\overline{R}_i|$ előre ismert, ezért nem kell minden Schreier-generátort megnéznünk.
Legfeljebb $\log |\overline{G}^{[i+1]}|$ elemet adtunk hozzá $S$-hez, mivel minden elem hozzáadása $\langle \overline{S} \cap \overline{G}^{[i+1]} \rangle$ elemszámát legalább megduplázta, így nem lesz túl nagy az új erős generátorrendszerünk.
$\overline{R}_{i+1}$ Schreier-fa ábrázolása $\overline{S}$ segítségével kiszámolható.

\subsection{Backtrack módszerek}
\label{subsec:permbt}
Ebben a részben legyen adott $G$ és legyen $\mathcal{P}$ egy tulajdonság, meg kell találnunk $G(\mathcal{P})$-t, vagyis $G$ azon elemeit, amik $\mathcal{P}$ tulajdonságúak.
Ehhez feltesszük, hogy minden $g \in G$-re el tudjuk dönteni, hogy $\mathcal{P}$ tulajdonságú-e.
Néhány tulajdonságra nincs ismert jobb algoritmus annál, mint hogy végigmegyünk $G$ elemein, és mindegyiket ellenőrizzük.
Gyakori eset azonban, hogy $G(\mathcal{P})$, ha nem üres, akkor $G$ egy részcsoportja, esetleg egy részcsoport mellékosztálya.
Például, ha egy elem vagy egy részhalmaz centralizátorát szeretnénk megtalálni, akkor egy részcsoportot keresünk.
Mellékosztályra példa, ha $a, b \in G$-re azokat a $g \in G$-ket keressük, amelyek $a$-t $b$-be konjugálják, vagyis amelyekre $a^g = g^{-1}ag = b$.
Gyakran elegendő azt eldöntenünk, hogy $G(\mathcal{P})$ üres-e, az előző példánál ez annak eldöntését jelenti, hogy $a$ és $b$ konjugáltak-e.
Ebben a részben feltesszük, hogy $G(\mathcal{P})$ egy részcsoport, egy részcsoport mellékosztálya vagy üres.

Legyen $B=(\beta_1,\dots,\beta_m)$ a $G \le \Sym(\Omega)$ csoport egy bázisa, és legyen $G = G^{[1]} \ge G^{[2]} \ge \dots \ge G^{[m]} \ge G^{[m+1]} = \{1\}$ az ehhez tartozó részcsoportlánc.
$\Omega$-n definiáljunk egy $\prec$-rendezést, amire $\beta_1 \prec \beta_2 \prec \dots \prec \beta_m$, valamint minden $\omega \in \Omega \setminus B$-re $\beta_m \prec \omega$.
$G$ minden elemét egyértelműen meghatározza az, hogy $B$ elemeit hova viszi, így a báziselemek képeinek sorozátaval azonosíthatjuk a csoportelemeket.
Ez az azonosítás egy rendezést is indukál $G$-n, $g, h \in G$-re $g \prec h$ akkor és csak akkor teljesül, ha létezik $i \in \{1,\dots,m\}$, amire minden $j < i$-re $\beta_j^g = \beta_j^h$, valamint $\beta_i^g \prec \beta_i^h$.
A báziselemek képeinek meghatározásával szeretnénk csoportelemeket keresni.
$(\beta_1,\dots,\beta_l)$ lehetséges képei $G^{[l+1]}$ mellékosztályait határozzák meg.
Ezeknek a képeknek a halmazát jelöljük $\mathcal{T}_l$-lel.
Legyen $\mathcal{T} = \bigcup_{i=1}^m \mathcal{T}_i$, ezen nézzük a tartalmazást, mint természetes részbenrendezést, amivel $\mathcal{T}$-t egy keresőfának tekinthetjük.
$\mathcal{T}$ gyökere (a $()$ sorozat) $G$-nek felel meg, a levelei (az $m$-hosszú sorozatok) a csoportelemeknek.
Jelöljük $t \in \mathcal{T}$-re $\mathcal{T}(t)$-vel a $t$-ből induló részfát.
Legyen $\varphi : \mathcal{T} \to P(G)$, amire $\varphi((\gamma_1,\dots,\gamma_l)) = \left\{ g \in G \mid \beta_i^g = \gamma_i \quad \forall 1 \le i \le l \right \}$.
Az algoritmusunk ezt a keresőfát járja be mélységi bejárással (backtrack), így keressük meg $G(\mathcal{P})$-t.
Amikor $t = (\gamma_1,\dots,\gamma_{l-1}) \in \mathcal{T}_{l-1}$-nél tartunk a bejárásban, akkor $\gamma_l$-nek nem kell minden $\Omega$-beli elemet végigpróbálnunk,
hanem elég azokat, amikre $\varphi((\gamma_1,\dots,\gamma_l))$ nem üres.
Mivel $\varphi(t)$ $G^{[l]}$-nek egy mellékosztálya, ezért ha veszünk egy tetszőleges $g \in \varphi(t)$ csoportelemet, akkor a szóbajövő $\gamma_l$-ek
azok, amikre $\gamma_l \in (\beta_l^{G^{[l]}})^g$.
Ez a bejárás a csoportelemeket mellesleg $\prec$-lexikografikus sorrendben találja meg.
Ha menet közben $t \in \mathcal{T}$-nél tartunk és $\varphi(t) \cap G(\mathcal{P})$-ről be tudjuk látni, hogy vagy üres, vagy hogy már minden elemét ismerjük, akkor a mélységi bejárást attól a ponttól nem kell mélyebbre folytatnunk.
Akkor lesz hatékony az algoritmusunk, ha minél nagyobb részfákat el tudunk így dobni.
Vannak a $\mathcal{P}$ tulajdonságtól független és attól függő elvek, amik alapján egy részfát eldobhatunk.
A következő lemma, az egyik legegyszerűbb $\mathcal{P}$-től független ilyen elvet mutatja be.
\begin{lemma}
\label{thm:permbtcrit1}
Tegyük fel, hogy $G(\mathcal{P}) \le G$, a $K = G(\mathcal{P}) \cap G^{[l+1]}$ részcsoportot már ismerjük valamilyen $l \in \{1,\dots,m\}$-re és hogy most valamelyik $t \in \mathcal{T}_l$-re $\mathcal{T}(t)$ belsejében vagyunk.
Ha találunk egyetlen $g \in \varphi(t) \cap G(\mathcal{P})$-t, akkor $\varphi(t) \cap G(\mathcal{P}) \subseteq \langle g, K \rangle \subseteq G(\mathcal{P})$, így kihagyhatjuk $\mathcal{T}(t)$ maradék részét.
\end{lemma}
\begin{proof}
Minden $h \in \varphi(t) \cap G(\mathcal{P})$ a $Kg$ mellékosztályban található.
\end{proof}
A következő lemmához tegyük fel, hogy ismerünk olyan $K, L \le G$ részcsoportokat, amikre igaz, hogy minden $g \in G$-re $g \in G(\mathcal{P})$ akkor és csak akkor, ha a kettős mellékosztály $KgL \subseteq G(\mathcal{P})$.
Ha $G(\mathcal{P})$ részcsoport, akkor tekinthetjük $K$-nak és $L$-nek $G(\mathcal{P})$ már ismert részcsoportját.
$P$-től függően azonban lehet, hogy más $K$-t meg $L$-et is választhatunk.
Például, ha $P$ azon elemekre teljesül, amik $a$-t $b$-be konjugálják, akkor választhatjuk $K=\langle a \rangle$-t és $L=\langle b \rangle$-t.
Mivel minden $KgL$ kettős mellékosztálynak elég egyetlen elemét megtalálnunk, és a bejárás $\prec$-lexikografikus sorrendben találja meg az elemeket, ezért a $\mathcal{T}(t)$ részfát kihagyhatjuk, ha tudjuk, hogy
egyik $g \in \varphi(t)$ sem lehet a saját $KgL$ kettős mellékosztályának első eleme.
Sajnos egy kettős mellékosztály első elemét megtalálni NP-nehéz probléma \cite{Luk93}, ezért kicsit másképp kell csinálnunk, erre való a következő 3 lemma.

\begin{lemma}
\label{thm:permbtcrit2}
Legyen $t = (\gamma_1, \dots, \gamma_l) \in \mathcal{T}$.
Ha $g \in \varphi(t)$ a $KgL$ kettős mellékosztály lexikografikusan első eleme, akkor $\gamma_l$ a $\gamma_l^{L_{(\gamma_1,\dots,\gamma_{l-1})}}$ orbit első eleme.
\end{lemma}
\begin{proof}
Indirekten tegyük fel, hogy van olyan $\gamma \in \gamma_l^{L_{(\gamma_1,\dots,\gamma_{l-1})}}$, amire $\gamma \prec \gamma_l$.
Legyen $h$ az az $L_{(\gamma_1,\dots,\gamma_{l-1})}$-beli elem, amire $\gamma_l^h = \gamma$.
Ilyenkor $gh \in KgL$ és $gh \prec g$, tehát ellentmondásra jutottunk.
\end{proof}
Eszerint a lemma szerint, amikor a $t' = (\gamma_1, \dots, \gamma_{l-1}) \in \mathcal{T}_{l-1}$ részfát nézzük, akkor elég $L_{(\gamma_1,\dots,\gamma_{l-1})}$ orbitjainak első elemeivel folytatnunk.
Ahhoz, hogy az orbitokat kiszámoljuk $L$-ben bázist kell váltanunk, azaz $L$-nek egy $(\gamma_1,\dots,\gamma_{l-1})$-gyel kezdődő bázishoz tartozó erős generátorrendszerét kell kiszámolnunk.
Emiatt ez a kritérium nem annyira gyorsan ellenőrizhető, sok implementációban csak kis $l$-ekre alkalmazzák.

Az előző lemma egy általánosabb alakja a következő:
\begin{lemma}
\label{thm:permbtcrit3}
Tegyük fel, hogy $\beta_l \in \beta_k^{K_{(\beta_1,\dots,\beta_{k-1})}}$ valamilyen $k \le l$-re.
Legyen $t = (\gamma_1, \dots, \gamma_l) \in \mathcal{T}$.
Ha $g \in \varphi(t)$ a $KgL$ kettős mellékosztály lexikografikusan első eleme, akkor $\gamma_k \preceq \min_{\prec}(\gamma_l^{L_{(\gamma_1,\dots,\gamma_{k-1})}})$.
\end{lemma}
\begin{proof}
Legyen $h_1$ az a $K_{(\beta_1,\dots,\beta_{k-1})}$-beli elem, amire $\beta_k^{h_1} = \beta_l$.
Indirekten tegyük fel, hogy van olyan $\gamma \in \gamma_l^{L_{(\gamma_1,\dots,\gamma_{k-1})}}$, amire $\gamma \prec \gamma_k$.
Legyen $h_2$ az az $L_{(\gamma_1,\dots,\gamma_{k-1})}$-beli elem, amire $\gamma_l^{h_2} = \gamma$.
$h_1gh_2 \in KgL$ és $h_1gh_2 \prec g$, mivel $i < k$-ra $\beta_i^{h_1gh_2} = \beta_i^{gh_2} = \gamma_i^{h_2} = \gamma_i = \beta_i^g$,
valamint $\beta_k^{h_1gh_2} = \beta_l^{gh_2} = \gamma_l^{h_2} = \gamma \prec \gamma_k = \beta_k^g$.
Ellentmondásra jutottunk.
\end{proof}
Ez $k=l$ esetén az ezelőtti lemmát adja vissza, $k < l$-re pedig azt jelenti, hogy $\beta_l \in \beta_k^{K_{(\beta_1,\dots,\beta_{k-1})}}$ esetén csak azok a $t = (\gamma_1, \dots, \gamma_l) \in \mathcal{T}_l$-ek
érdekesek, amikre $\gamma_l \succ \gamma_k$.

\begin{lemma}
\label{thm:permbtcrit4}
Legyen $s = \left| \beta_l^{K_{(\beta_1,\dots,\beta_{l-1})}} \right|$, azaz $K$ $l$-edik fundamentális orbitjának mérete.
Legyen $t = (\gamma_1, \dots, \gamma_l) \in \mathcal{T}$.
Ha $g \in \varphi(t)$ a $KgL$ kettős mellékosztály lexikografikusan első eleme, akkor $\gamma_l$ nem lehet a $\gamma_l^{G_{(\gamma_1,\dots,\gamma_{l-1})}}$ orbit utolsó $s-1$ eleme közt.
\end{lemma}
\begin{proof}
A $\Gamma = \left\{ \beta_l^{hg} \mid h \in K_{(\beta_1,\dots,\beta_{l-1})} \right\}$ halmaz $s$ elemű és $\gamma_l = \beta_l^g \in \Gamma$.
Minden $\gamma = \beta^{hg} \in \Gamma$ $G_{(\gamma_1,\dots,\gamma_{l-1})}$-nek ugyanahhoz az orbitjához tartozik, mivel $\gamma^{g^{-1}h^{-1}g} = \gamma_l$, ahol $g^{-1}h^{-1}g \in G_{(\gamma_1,\dots,\gamma_{l-1})}$.
$hg \in KgL$ és $g$ minimalitása miatt $\gamma_l = \min_{\prec} \Gamma$, ezért legalább $s-1$ elem jön késöbb $KgL$-ben.
\end{proof}
Emiatt amikor a $t' = (\gamma_1, \dots, \gamma_{l-1}) \in \mathcal{T}_{l-1}$ részfát nézzük, akkor $G_{(\gamma_1,\dots,\gamma_{l-1})}$ minden orbitjának utolsó $s-1$ elemét kihagyhatjuk.

Most nézzünk $\mathcal{P}$-specifikus kritériumokat.
Nézzük először a centralizátor problémát, legyen adott $c \in G$, $C_G(c) = G(\mathcal{P})$-t kell megtalálnunk.
Mivel $C_G(c) = C_G(\langle c \rangle)$, ezért minden $\omega \in \Omega, g \in C_G(c), k \in \mathbb{Z}$-re igaz, hogy $(\omega^{c^k})^g = (\omega^g)^{c^k}$.
Ez két dolgot jelent, egyrészt hogy ha $\omega$ képét meghatároztuk már, akkor az a $\omega^{c^k}$ alakú elemek képeit is meghatározza,
másrészt pedig hogy ha $\omega$ egy $l$-hosszú ciklus része $c$-ben, akkor $\omega^g$ is csak egy $l$-hosszú ciklus része lehet $c$-ben.
Ezeket a feltételeket akkor tudjuk hatékonyan kihasználni, hogyha olyan bázisban írjuk fel az erős generátorrendszert, amiben $c$ egyes ciklusainak az elemei egymást követik.
Így tudunk $\omega, \omega^c, \omega^{c^2}, \dots$ közül minél többől báziselemet csinálni, tehát így lesz minél több $\mathcal{T}(t)$ eldobható.

Második példának vegyünk $\Delta \subseteq \Omega$-t és legyen $G(\mathcal{P}) = G_{\Delta}$, vagyis a $\Delta$ halmaz stabilizátorát keressük.
Ehhez először az $\Omega$ elemeinek egy olyan rendezése szerint csináljuk meg a bázist és az erős generátorrendszert, amiben a $\Delta$-beli elemek vannak elől.
Legyen $\beta_k$ az első $\Omega \setminus \Delta$-beli báziselem, ilyenkor tudjuk, hogy $G^{[k]} = G_{(\Delta)} \le G(\mathcal{P})$.
A \ref{thm:permbtcrit1}. lemma szerint $\varphi(t) \cap G(\mathcal{P}) = \emptyset$ vagy $\varphi(t) \subseteq G(\mathcal{P})$, minden $t \in \mathcal{T}_{k-1}$-re.
Így elég csak az első $k-1$ báziselem képét vizsgálnunk.

Harmadik példának nézzük a konjugáltság-ellenőrzés esetét, adott $a, b \in G$, $G(\mathcal{P}) = \{g \in G \mid a^g = b \}$.
Ebben a problémában $G(\mathcal{P})$ vagy üres, vagy $C_G(a)$ egy jobboldali mellékosztálya, ezért hasonlít ez az eset a centralizátor-problémára, ahol hasonló feltételek jöttek ki.
$g$-nek $a$ ciklusait $b$ ugyanolyan hosszú ciklusaihoz kell rendelnie, így ha $a$ egy ciklusának egy elemének a képét meghatározzuk, az az egész ciklus képét is megmondja.
Itt is speciális bázisban érdemes felírni az erős generátorrendszert, mégpedig olyanban amiben $a$ ciklusainak elemei egymást követik.

Másik fontos itt említendő probléma a következő:
Legyen adottak a $G, H \le \Sym(\Omega)$ csoportok, a $G\cap H$ csoportnak szeretnénk egy generátorrendszerét meghatározni.
Feltehetjük, hogy $|G| \le |H|$.
Először számoljuk ki $G$-nek egy $B = \{\beta_1, \dots, \beta_m\}$ bázisát, és definiáljuk $\mathcal{T}$-t és $\varphi$-t, mint eddig.
Számoljuk ki $H$-nak egy olyan bázisát, ami $B$-vel kezdődik és egy ehhez tartozó erős generátorrendszert.
Legyen $\varphi' : \mathcal{T} \to P(H)$ $\varphi$-hez analóg módon definiálva $H$-ra.
Ha $t = (\gamma_1, \dots, \gamma_{l-1}) \in \mathcal{T}_{l-1}$ részfáját nézzük, akkor csak azokkal a $\gamma_l$ elemekkel kell folytatnunk,
amik nem csak $G$-ben lehetséges folytatások, hanem $H$-ban is, vagyis a $\gamma_l \in (\beta_l^{G^{[l]}})^g \cap (\beta_l^{H^{[l]}})^h$ alakúakat,
ahol $g \in \varphi(t)$ és $h \in \varphi'(t)$.
Így minden $\mathcal{T}_m$-beli elem amihez eljutunk egy olyan $G$-beli elemet reprezentál, ami $H$-ban is benne van,
vagyis a menetközben felírt generátorelemei $G(\mathcal{P})$-nek azok pont $G\cap H$-nak lesznek a generátorelemei.

\subsection{Policiklikus csoportok}
\label{subsec:policiklikus}
Policiklikus csoportok

\subsubsection{Policiklikus prezentáció}
PC-relációk, elemek prezentálása

\subsubsection{Algoritmusok}
Orbit-stabilizátor módszerek, homomorfizmusok, részcsoportok, centralizátor, konjugáltosztályok, ...

\clearpage
