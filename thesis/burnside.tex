\subsection{Burnside-Dixon-Schneider algoritmus}
\label{subsec:bds}
Az első algoritmust karaktertáblázatok számítására Burnside írta le \cite{Bur11}-ben. Azóta is az ő módszere, és annak átalakított változatai
az egyetlen minden csoportra alkalmazható ismert algoritmus.
Ebben az alfejezetben ismertetjük az eredeti algoritmus ötleteit, illetve pár változtatást.
\todo{bekezdés: Mathematica kód}

\subsubsection{Burnside eredeti algoritmusa}
\label{subsubsec:bdseredeti}
Legyen $G$ tetszőleges véges csoport, amit multiplikatívnak tekintünk, egységelemét szokásosan $1$-gyel jelöljük.
Jelöljuk $G$ konjugáltosztályainak számát $r$-rel, az osztályokat $C_1, C_2, \dots, C_r$-rel, ezeknek egy-egy reprezentáns elemét $g_1, g_2, \dots, g_r$-rel.
Legyen $i=1, 2, \dots, r$-re $h_i=|C_i|=|G:C_G(C_i)|$, a konjugáltosztályok elemszáma.
Természetesen választhatjuk $g_1$-et $1$-nek, így $h_1=1$.

Feltételezzük, hogy az \ref{sec:csoport}. fejezetben leírtak szerint $g_i$-k, $h_i$-k már ki vannak számolva.
Azt is fel kell tennünk, hogy minden $g\in G$-re gyorsan meg tudjuk állapítani, hogy melyik konjugáltosztályba tartozik.
Kisebb csoportok esetében, ha van elég memória, érdemes minden $g$-re előre kiszámolnunk ezt.

Jelöljük $G$ irreducibilis karaktereit $\chi^1, \chi^2, \dots, \chi^r$-rel, és a rövidség kedvéért legyen $\chi^i_j=\chi^i(g_j)$.
Legyen $d_i=\chi^i_1$, azaz $\chi^i$ foka.

$1 \le j, k, l \le r$-re legyen $c_{jkl}$ azon elempárok száma, ahol az egyik elem $C_j$-beli, a másik elem $C_k$-beli és a szorzatuk $g_l$.
Ismert, hogy $c_{jkl}$ független $g_l$ megválasztásától, valamint hogy teljesül a következő egyenlőség $\forall 1 \le i, j, k \le r$-re:
\begin{equation}
\label{eq:bdscjkl1}
\frac{h_j\chi^i_j}{d_i}\frac{h_k\chi^i_k}{d_i}=\sum_{l=1}^r c_{jkl}\frac{h_l\chi^i_l}{d_i}
\end{equation}
Legyen $M_j$ az az $r\times r$-es mátrix, aminek $(k,l)$-edik eleme $c_{jkl}$, ezeket könnyen ki tudjuk számolni.
Legyen $v^i=[h_1\chi^i_1/d_i, h_2\chi^i_2/d_i, \dots, h_r\chi^i_r/d_i]^T$.
Ha $i, j$-t lerögzítjük, akkor ezekkel a jelölésekkel a (\ref{eq:bdscjkl1}) egyenlet átírható a következő alakba:
\begin{equation}
\label{eq:bdscjkl2}
\frac{h_j\chi^i_j}{d_i}v^i=M_j v^i
\end{equation}
Vagyis $v^i$ jobboldali sajátvektora $M_j$-nek $\forall i, j$-re.
Van tehát $r$ darab közös sajátvektorunk minden $M_j$-hez, amik közül semelyik kettőhöz sem tartozhat minden $M_j$ esetén ugyanazok a sajátértékek,
így ezeket a sajátvektorokat $M_j$-kből konstansszorzó erejéig egyértelműen meghatározhatjuk lineáris algebrai ismereteink alapján.
A helyes konstansszorzókat könnyen megkaphatjuk, hiszen $v^i_1=h_1\chi^i_1/d_i=1 d_i/d_i=1$, így $M_j$-kből meg tudjuk állapítani $v^i_j$-ket.

Kérdés, hogy $v^i_j$-k ismeretében hogyan állapíthatjuk meg $\chi^i_j$-ket.
$h_j$-ket ismerjük, tehát elegendő $d_i$-k kiszámolása.
Szintén ismert, hogy a komplex-értékű osztályfüggvények terén értelmezhető egy természetes skalárszorzat a következőképpen:
\begin{equation}
\label{eq:bdsscalar}
\langle \alpha, \beta \rangle = \frac{1}{|G|}\sum_{g\in G}\alpha(g)\overline{\beta(g)} = \frac{1}{|G|}\sum_{j=1}^r h_j \alpha(g_j) \overline{\beta(g_j)}
\end{equation}
A karakterekre vonatkozó ortogonális relációk alapján $\langle \chi^i, \chi^j \rangle = \delta_{ij}$, ami alapján
\begin{equation}
\label{eq:bdsdi1}
1 = \langle \chi^i, \chi^i \rangle = \frac{1}{|G|}\sum_{j=1}^r h_j |\chi^i_j|^2 = \frac{1}{|G|}\sum_{j=1}^r h_j \left|\frac{d_i v^i_j}{h_j}\right|^2 = 
\frac{d_i^2}{|G|}\sum_{j=1}^r \frac{|v^i_j|^2}{h_j}
\end{equation}
Ezalapján $d_i$ kifejezhető, tehát sikerült $\chi^i_j$-ket kiszámolnunk.

\noindent
Összefoglalva az algoritmust:
\begin{enumerate}
\item Kiszámoljuk $r, C_i, g_i, h_i, M_i$-ket
\item Kiszámoljuk $M_i$-knek az $r$ darab közös sajátvektorát, majd ezekből $v^i_j$-ket
\item Kiszámoljuk $d_i$-ket, és végül ebből $\chi^i_j$-ket
\end{enumerate}

\subsubsection{Schneider első módosítása}
\label{subsubsec:bdssch1}
Ezt a módosítást Schneider írta le \cite{Sch90}-ben.
Nem annyira a sebességén javít az algoritmusnak, hanem inkább egyszerűbbé teszi azt, ezért ezt vesszük előre,
annak ellenére is, hogy a \ref{subsubsec:bdsdixon}. részben leírt módosítást Dixon már jóval előtte kitalálta.

Jelöljük $1\le j \le r$-re $j'$-vel azt a szintén $[1, r]$-beli számot, amire $g_j^{-1}$ és $g_{j'}$ konjugáltak,
vagyis amire $C_{j'}=C_j^{-1}$. Mivel $|C_l|=h_l$, így azon $(x, y, z)\in C_j \times C_k \times C_l$ hármasok száma,
amire $xy=z$ egyenlő $h_l c_{jkl}$, de szintén egyenlő azon hármasok számával, amikre $x^{-1}z=y$, vagyis $h_l c_{jkl}=h_k c_{j'lk}$.
Ezt a (\ref{eq:bdscjkl1}) egyenlet jobb oldalába beírva, majd $j$-t és $j'$-t felcserélve kapjuk, hogy
\begin{equation}
\label{eq:bdssch1}
\frac{h_{j'}\chi^i_{j'}}{d_i}\chi^i_k=\sum_{l=1}^r c_{jlk}\chi^i_l
\end{equation}
Vagyis a $[\chi^i_1, \chi^i_2, \dots, \chi^i_r]$ sorvektor baloldali sajátvektora $M_j$-nek.

Konstansszorzók erejéig most is egyértelműen meghatározhatjuk $M_j$-kből ezeket a vektorokat, most viszont a konstansszorzó megállapításához
kell használnunk a karakterek ortonormáltságát. Tegyük fel, hogy találtunk $r$ darab közös sajátvektort, jelöljük ezeket $w^i$-vel.
A keresett szorzó $d_i/w^i_1$, mert az első elemnek $d_i$-nek kell lennie.
\begin{equation}
\label{eq:bdssch2}
1 = \langle \chi^i, \chi^i \rangle = \frac{1}{|G|}\sum_{j=1}^r h_j |\chi^i_j|^2 = \frac{1}{|G|}\sum_{j=1}^r h_j \left|\frac{d_i}{w^i_1} w^i_j\right|^2 = 
\frac{d_i^2}{|G| |w^i_1|^2} \sum_{j=1}^r h_j |w^i_j|^2
\end{equation}
Ezalapján $d_i$-ket megállapíthatjuk, amiből meg $\chi^i_j$-ket megkapjuk.


\subsubsection{Dixon módosítása}
\label{subsubsec:bdsdixon}
Jelöljük $G$ exponensét, vagyis az elemrendek legkisebb közös többszörösét $e$-vel.
Ismert, hogy $\chi^i_j$-k algebrai egészek, sőt éppen $d_i$ darab komplex $e$-edik egységgyök összege,
vagyis az algoritmustól elvárható kell, hogy legyen, hogy a karaktertáblázat elemeit ne csak közelítő értékekkel, hanem pontos értékekkel kapjuk meg.
Az eddig leírt módszer azonban csak közelítésre volt alkalmas, ugyanis az $M_j$ mátrixok közös sajátértékeit csak úgy tudjuk meghatározni.
Erre Dixon talált egy megoldást, amit \cite{Dix67}-ben írt le. Az alapötlet az, hogy válasszunk egy alkalmas $p$ prímet, minden számítást $\mathbb{F}_p$
felett végezzünk el, majd a kapott eredményt emeljük vissza $\mathbb{C}$-be.

Jelöljünk egy $e$-edik komplex primitív egységgyököt $\zeta$-val. Minden karaktertáblázatbeli elem $\mathbb{Z}[\zeta]$-beli.
$h_j\chi^i_j/d_i$-ről is ismert, hogy algebrai egész, tehát $h_j\chi^i_j/d_i \in \mathbb{Q}[\zeta] \cap \mathbb{A} = \mathbb{Z}[\zeta]$, így a
(\ref{eq:bdscjkl1}), (\ref{eq:bdssch1}) egyenletek is $\mathbb{Z}[\zeta]$-beliek.

Válasszunk egy olyan $p$ prímet, amire $e|p-1$ és $p > 2 d_i$ minden $1\le i\le r$-re. Mivel $d_i$-ket nem ismerjük még, de tudjuk, hogy 
$\sum_{i=1}^r d_i^2 = |G|$, ezért $p > 2\lfloor \sqrt{|G|} \rfloor$ elegendő. Mivel $e|p-1$, ezért $\mathbb{F}_p$-ben van $e$-rendű elem, jelöljünk
egy ilyet $\omega$-val. Így adódik egy $\Theta : \mathbb{Z}[\zeta] \to \mathbb{F}_p$ gyűrűhomomorfizmus, a következő hozzárendeléssel:
\begin{equation*}
\sum_{i=0}^{e-1} a_i \zeta^i \mapsto \sum_{i=0}^{e-1} a_i \omega^i
\end{equation*}
$\Theta$-t alkalmazva az (\ref{eq:bdssch1}) egyenletre $\mathbb{F}_p$ feletti egyenletrendszert kapunk,
tehát a $[\Theta(\chi^i_1), \Theta(\chi^i_2), \dots, \Theta(\chi^i_r)]$ sorvektorok $\mathbb{F}_p$ felett baloldali sajátvektorai $\Theta(M_j)$-knek.
Be kell látnunk, hogy ezek a sorvektorok lineárisan függetlenek $\mathbb{F}_p$ felett.
Legyen $X$ az a mátrix, aminek $(i, j)$-edik eleme $\chi^i_j$, vagyis $X$ maga a karaktertáblázat.
Legyen $Y$ az a mátrix, aminek $(j, i)$-edik eleme $h_j\chi^i_j$.
A karakterek ortonormáltsága miatt $XY=|G|I_r$.
Erre alkalmazva $\Theta$-t azt kapjuk, hogy $\Theta(X)\Theta(Y)=\Theta(XY)=\Theta(|G|I_r)=\Theta(|G|)I_r$.
Mivel $|G|$ minden $q$ prímosztójára $q|e$, ezért $q|p-1$, vagyis $\Theta(|G|)\ne 0$, tehát $\Theta(X)$ sorai tényleg lineárisan függetlenek $\mathbb{F}_p$ felett.
Az $i_1$ és $i_2$-edik sorokhoz tartozó sajátértékek nem mind egyezhetnek meg, ugyanis akkor az a két sor lineárisan összefüggne.
Így a $\Theta(M_j)$ mátrixok $\mathbb{F}_p$ felett is skalárszorzó erejéig egyértelműen meghatározzák a közös sajátvektoraikat.

Tehát miután kiszámoltuk $\Theta(M_j)$-ket, kiszámolhatjuk a karakterisztikus polinomjaikat, amit faktorizálva megkapjuk a sajátértékeket, amiből végül meghatározhatjuk a baloldali sajátvektorokat.
Ezeket a műveleteket véges testek felett mind könnyen el tudjuk végezni, a részletekbe most nem megyünk bele.

A kapott sajátvektorokat normalizálhatjuk úgy, hogy megszorozzuk az első elemük inverzével, vagyis azokat a sajátvektorokat nézzük, amiknek az első eleme 1.
Ezeket a sajátvektorokat jelöljük $v^i$-vel. Mivel $\Theta(\chi^i_1)=\Theta(d_i)=d_i$, így $\Theta(\chi^i_j)=d_i v^i_j$, tehát $d_i$-t kell meghatároznunk.
Az ismert $\chi^i(g^{-1})=\overline{\chi^i(g)}$ azonosságból kapjuk, hogy
\begin{equation}
\label{eq:bdssch3}
1 = \langle \chi^i, \chi^i \rangle = \frac{1}{|G|}\sum_{j=1}^r h_j \chi^i_j \overline{\chi^i_j} = \frac{1}{|G|}\sum_{j=1}^r h_j \chi^i_j \chi^i_{j'}
\end{equation}
$|G|$-vel szorozva, majd $\Theta$-t ráalkalmazva
\begin{equation}
\label{eq:bdssch4}
|G| \equiv \sum_{j=1}^r h_j \Theta(\chi^i_j) \Theta(\chi^i_{j'}) \equiv d_i^2 \sum_{j=1}^r h_j v^i_j v^i_{j'} \pmod{p}
\end{equation}
Ebből $d_i^2$ $p$-s maradékát megkapjuk, amiből $d_i$ egyértelmű, mert $0<d_i<p/2$ lehet csak.

$\Theta(\chi^i_j)$-ket már tudjuk, az algoritmus utolsó lépése, hogy visszakapjuk $\chi^i_j$-ket, vagyis a komplex értékeket.
Mivel $\chi^i_j$ $d_i$ darab $\zeta$-hatvány összege, ezért léteznek olyan $0\le m_{ijk}\le d_i < p$ számok, amikre:
\begin{equation}
\label{eq:bdsmijk1}
\chi^i_j = \sum_{k=0}^{e-1} m_{ijk}\zeta^k
\end{equation}
Jelöljük $j(l)$-lel annak a konjugáltosztálynak a számát, amiben $g_j^l$ van. Az előző egyenletből $\chi^i_{j(l)}$-t is ki tudjuk fejezni, mégpedig:
\begin{equation}
\label{eq:bdsmijk2}
\chi^i_{j(l)} = \sum_{k'=0}^{e-1} m_{ijk'}\zeta^{k'l}
\end{equation}
Ha a következő összeget tekintjük, majd ezt belehelyettesítjük:
\begin{equation*}
\label{eq:bdsmijk3a}
\frac{1}{e}\sum_{l=0}^{e-1} \chi^i_{j(l)} \zeta^{-kl} =
\frac{1}{e}\sum_{l=0}^{e-1} \sum_{k'=0}^{e-1} m_{ijk'}\zeta^{(k'-k)l} =
\frac{1}{e}\sum_{k'=0}^{e-1} m_{ijk'} \sum_{l=0}^{e-1} \zeta^{(k'-k)l} =
\end{equation*}
\begin{equation}
\label{eq:bdsmijk3b}
= \frac{1}{e}\sum_{k'=0}^{e-1} m_{ijk'} e \delta_{kk'} = m_{ijk}
\end{equation}
$\Theta$-t ráalkalmazzuk az egyenletre:
\begin{equation}
\label{eq:bdsmijk4}
m_{ijk} = \Theta(m_{ijk}) = \Theta(e)^{-1} \sum_{l=0}^{e-1} \Theta(\chi^i_{j(l)}) \omega^{-kl}
\end{equation}
Ezzel $\Theta(\chi^i_j)$-kből ki tudtuk fejezni $\chi^i_j$-ket, vagyis készen vagyunk.
\todo{rendes példa: Dixonra}

\subsubsection{Schneider második módosítása}
\label{subsubsec:bdssch2}
Nem esett szó eddig arról, hogy hogyan lehet az $M_j$ mátrixokat kiszámolni.
Egyszerre $M_j$ egy oszlopát (legyen most a $l$-edik) tudjuk megadni, mégpedig úgy hogy $\forall x \in C_j$-re kiszámoljuk
$y=x^{-1}g_l$-t és azt, hogy $y$ melyik konjugáltosztályba esik.
$M_j$ $(k,l)$-edik eleme azon $x$-ek száma, amikre $y\in C_k$.
Általában az algoritmus futásának legnagyobb részét ez a számolás teszi ki, ezért arra kell törekednünk, hogy minél kevesebb oszlop kiszámolásával meghatározhatóak legyenek a sajátvektorok.
Vegyük észre, hogy $M_j$ első oszlopának $j'$-edik eleme $h_j$, az összes többi eleme $0$, így az első oszlopokat már ismerjük.
$M_1$-et is ismerjük, hiszen annak $(k,l)$-edik eleme $\delta_{kl}$, vagyis $M_1=I_r$.

Ebben a részben a sajátvektorok kiszámításához mutatunk egy gyorsabb módszert, amihez nem kell mindegyik $M_j$ mátrixot teljesen kiszámolnunk.
Nevezzük karakteraltereknek $\mathbb{F}_p^r$ azon altereit, amelyeket néhány karakter generál, vagyis a $\langle \Theta(\chi^i) | i \in I \rangle$ alakban felírhatóakat, ahol $I \subseteq \{1,\dots,r\}$.
A sajátvektorokat a következőképpen szeretnénk kiszámolni:
kiindulunk egyetlen karakteraltérből (a teljes $\mathbb{F}_p^r$-ről), és minden lépésben egy legalább két dimenziós karakteralteret egy $M_j$ segítségével felbontunk több kisebb dimenziós karakteraltér direktösszegére,
úgy hogy a karakteralternek nézzük a metszetét $M_j$ sajátaltereivel.
Ezt addig ismételjük, amíg csak 1-dimenziós karakteraltereink maradnak, ezek a keresett közös sajátvektorok lesznek.
\todo{jo szo kell: oszlop-redukalt bazis? (angolul: row reduced basis, row echelon basis, echelonized basis)}
Egy alteret mindig egy oszlop-redukált bázisként tekintünk és tárolunk, ami azt jelenti, hogy a generátorvektorokat egy mátrix soraiként leírva olyat kapunk,
amire teljesül, hogy minden sor első nemnulla eleme $1$, valamint ha az $i$-edik sorban a $j$-edik elem az első nemnulla elem, akkor $\forall i'>i$ és $\forall j'\le j$-re az $(i',j')$ helyeken 0 van.
Egy tetszőlegesen megadott bázisból Gauss-eliminációval, majd az üres sorok törlésével oszlop-redukált bázist kaphatunk.

Ahhoz, hogy eldöntsük, hogy melyik $M_j$-vel érdemes próbálkozni egy adott $V$ karakteraltér felbontásánál, a következő lemmára lesz szükségünk:
\begin{lemma}
\label{thm:bdsmjfix}
Legyen $b_1,b_2,\dots,b_s$ a $V$ karakteraltér oszlop-redukált bázisa.
Akkor és csak akkor esik $V$ a $\Theta(M_j)$ mátrix egyetlen baloldali sajátalterébe, ha minden olyan $V$-beli vektorra, aminek az első koordinátája $0$, teljesül az,
hogy jobbról szorozva $\Theta(M_j)$-vel a kapott vektornak is $0$ az első koordinátája.
\end{lemma}
\begin{proof}
$V$ karakteraltér, tehát van benne olyan vektor, aminek első koordinátája nemnulla, így $b_1$ első koordinátája $1$.
Ha $V$ egy sajátaltérbe esik, akkor minden $V$-beli vektor sajátvektor, vagyis ha egy vektor első koordinátája $0$, akkor a szorzatnak is.
A másik irány bizonyításához tegyük fel indirekten, hogy $V$-be belemetsz két különböző sajátértékhez ($\lambda_1, \lambda_2$) tartozó sajátaltér is.
Mivel $V$ és egy sajátaltér metszete karakteraltér, ezért van olyan vektor a metszetben, aminek első koordinátája $1$. Legyen egy $\lambda_i$-hez tartozó ilyen vektor $v_i$.
$b_1=v_i+w_i$ alakba írható, ahol $w_i$ első koordinátája $0$. Ezt jobbról $\Theta(M_j)$-vel szorozva azt kapjuk, hogy
$b_1\Theta(M_j)=v_i\Theta(M_j)+w_i\Theta(M_j)=\lambda_i v_i+w_i\Theta(M_j)$. Ha csak az első koordinátákat nézzük, akkor azt kapjuk, hogy $b_1\Theta(M_j)$ első koordinátája $\lambda_i$.
Ezt $i=1,2$-re is megkaptuk, vagyis $\lambda_1=\lambda_2$, ami ellentmondás.
\end{proof}
Ez azt jelenti, hogy egy $V=\langle b_1, b_2, \dots, b_s \rangle$ karakteralteret $M_j$-vel akkor tudunk felbontani, ha van olyan $V$-beli vektor,
aminek első koordinátája $0$, de $\Theta(M_j)$-vel vett szorzatának első koordinátája nem $0$.
Ez ekvivalens azzal, hogy $\exists 2\le i\le s$, amire $b_i \Theta(M_j)$ első koordinátája nem $0$.
A szorzat első koordinátája $b_i$ és a mátrix első oszlopának skalárszorzata, és mivel tudjuk, hogy az első oszlopban pontosan egy nemnulla elem van, mégpedig a $j'$-edik helyen,
ezért az kell, hogy $b_i$ $j'$-edik koordinátája ne legyen $0$.
Összefoglalva, akkor tudjuk $V$-t felbontani $M_j$-vel, ha $\exists 2\le i\le s$, amire $b_i$ $j'$-edik koordinátája nem $0$.

Ha $V=\langle b_1, \dots, b_s \rangle$ oszlop-redukált alakban van megadva, $b_i$ első nemnulla koordinátája a $c_i$-edik,
akkor a $\Theta(M_j)$ mátrix hatását $V$-n ki tudjuk számolni csak a $c_1, \dots, c_s$-edik oszlopok ismeretével.
Ugyanis egy $V$-beli vektort egyértelműen meghatároznak a $c_1, \dots, c_s$-edik koordinátái, vagyis ha $\forall 1\le i\le s$-re $b_i\Theta(M_j)$-nek kiszámoljuk a $c_1, \dots, c_s$-edik koordinátáit,
akkor azok alapján a szorzatot fel tudjuk írni $b_i$-k lineáris kombinációjaként, vagyis ezek alapján az oszlopok alapján fel tudunk írni egy olyan $s\times s$-es $A$ mátrixot,
ami a $b_i$ bázisban felírt $\Theta(M_j)$-vel való szorzásnak felel meg.
Ennek kell a sajátaltereit meghatároznunk.
Ehhez kiszámoljuk a karakterisztikus polinomját, azt $F_p$ felett faktorizálni tudjuk például a Cantor-Zassenhaus algoritmussal \cite{CZ81}, így megkaptuk a $\lambda_i$ sajátértékeket.
$A-\lambda_i I$ baloldali nulltere lesz az $i$-edik sajátaltér, amit az eredeti bázisba visszaírva megkapjuk $V$ és $\Theta(M_j)$ $i$-edik baloldali sajátalterének metszetét.

\subsubsection{További módosítások}
\label{subsubsec:bdstovabbi}
Ha Schneider második módosítását követjük, akkor segít az algoritmusnak ha előre ki tudunk számolni pár irreducibilis karaktert.
A triviális (csupa $1$) karaktert például mindig fel tudjuk használni.
Általában megéri, hogy a lineáris karaktereket, illetve azok $\Theta$-általi képeit előre kiszámoljuk.
Először nézzk meg, hogyan tudjuk kiszámolni ezeket Abel-csoportokra.
(Valójában Abel-csoportoknál csak ezeket kell kiszámolni, hiszen minden irreducibilis karakter lineáris.)
Ilyenkor minden elem egy önálló konjugáltosztályt alkot.
Ha $G$ egy ciklikus csoport ($G\simeq Z_n$), akkor minden irreducibilis karakter úgy néz ki, hogy a generátorelemhez hozzárendel egy $n$-edik komplex egységgyököt, a hatványaihoz pedig az egységgyök megfelelő hatványait.
Ha $G$ Abel-csoport, akkor a véges Abel-csoportok alaptétele szerint felírható prímhatványrendű ciklikusok direktszorzataként, így ezt a felbontást megkeresve a tényezők irreducibilis karaktereit ki tudjuk számolni,
majd ezek direktszorzataként megkapjuk $G$ irreducibilis karaktereit.
Abban az esetben ha $G$ nem kommutatív, $G$ abelizálását kell tekintenünk, vagyis a $G^{\mathop{ab}}=\faktor{G}{G'}=\faktor{G}{[G,G]}$ Abel-csoportnak kell meghatároznunk a karaktertábláját,
majd ezen karaktereket $G$-re visszaemelve megkaphatjuk $G$ lineáris irreducibilis karaktereit.
\todo{bds egyéb módosítások}
